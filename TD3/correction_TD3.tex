%
%   TMB-TD-2019.tex  - Tine Léon Matar
%
%%%%%%%%%%%%%%%%%%%%%%%%%%%%%%%%%%%%%%%%%%%%%%%%%%%%%%%%%%%%%%%%%%%%%%

\documentclass[a4paper,11pt]{article}

\usepackage{amsfonts,amsmath,amssymb,amscd,amstext,mathabx}
\usepackage{textcomp,multicol,enumitem,mdwlist}
\usepackage{pict2e,graphicx}

\usepackage[T1]{fontenc}
\usepackage[french]{babel}

\setlength{\textheight}{27cm}
\setlength{\textwidth}{18cm}
\setlength{\topmargin}{-2.8cm}
%\setlength{\leftmargin}{-.5cm}
\setlength{\oddsidemargin}{-1.05cm}
\setlength{\evensidemargin}{-1.05cm}
 
%\setcounter{page}{0}

\renewcommand{\arraystretch}{1.5}

\newcommand{\bitem}{\item[$\bullet$]}
\newcommand{\ds}{\displaystyle}

\newcommand{\R}{\mathbb R}
\newcommand{\C}{\mathbb C}
\newcommand{\N}{\mathbb N}
\newcommand{\Z}{\mathbb Z}
\newcommand{\Q}{\mathbb Q}
\newcommand{\dist}{\mathrm{dist}\,}
\newcommand{\Vect}{\mathrm{Vect}\,} 
\newcommand{\Lin}{\mathrm{Lin}\,} 
\newcommand{\calF}{{\cal F}}
\newcommand{\calM}{{\cal M}}
\newcommand{\Aire}{\mathrm{Aire}\,}
\newcommand{\ch}{\mathrm{ch}\,}
\newcommand{\sh}{\mathrm{sh}\,}
\renewcommand{\th}{\mathrm{th}\,}

\newcommand{\vi}{\vec{\mbox{\it \i}}\,}
\newcommand{\vj}{\vec{\mbox{\it \j}}\,} 
\newcommand{\vk}{\vec{\mbox{\it k}}\,} 
\newcommand{\ve}{\vec{e}}
\newcommand{\vu}{\vec{u}}
\newcommand{\vv}{\vec{v}}
\newcommand{\vw}{\vec{w}}
\newcommand{\vh}{\vec{h}}
\newcommand{\vx}{\vec{x}}
\newcommand{\vy}{\vec{y}}
\newcommand{\vXY}[2]{\left(\!\!\begin{array}{c} #1 \\ #2 \end{array}\!\!\right)}
\newcommand{\vXYZ}[3]{\left(\!\!\begin{array}{c} #1 \\ #2 \\ #3 \end{array}\!\!\right)}

\newcommand{\vA}{\overrightarrow{A}}
\newcommand{\vB}{\overrightarrow{B}}
\newcommand{\vE}{\overrightarrow{E}}
\newcommand{\vF}{\overrightarrow{F}}
\newcommand{\vV}{\overrightarrow{V}}
\newcommand{\vU}{\overrightarrow{U}}

\newcommand{\OXY}{\begin{picture}(30,20)(0,-5)
                  \put(-20,0){\vector(1,0){45}}
                  \put(17,-7){\footnotesize $x$}
                  \put(0,-20){\vector(0,1){45}}
                  \put(-7,17){\footnotesize $y$}
                  \end{picture}}
\newcommand{\OXYZ}{\begin{picture}(30,10)(0,0)
                   \put(10,20){\vector(-1,-2){20}}
                   \put(-16,-18){\footnotesize $x$}
                   \put(-20,0){\vector(1,0){45}}
                   \put(19,-7){\footnotesize $y$}
                   \put(0,-20){\vector(0,1){45}}
                   \put(-7,19){\footnotesize $z$}
                   \end{picture}}


\newenvironment{alphate}{\begin{enumerate}[label=\alph*)]}{\end{enumerate}}

\newtheorem{exo}{Exercice}
\newenvironment{exercice}{\begin{exo} \em}{\end{exo}}

\newtheorem{rep}{R\'eponses}
\newenvironment{reponse}{\begin{rep} \em}{\end{rep}}


%%%%%%%%%%%%%%%%%%%%%%%%%%%%%%%%%%%%%%%%%%%%%%%%%%%%%%%%%%%%%%%%%%%%%%

\begin{document}

%%%%%%%%%%%%%%%%%%%%%%%%%%%%%%%%%%%%%%%%%%%%%%%%%%%%%%%%%%%%%%%%%%%%%%%



\begin{center}
{\bf Correction TD 3 \quad--\quad \'EQUATIONS, IN\'EQUATIONS ET DOMAINE DE 
D\'EFINITION}
\end{center}
\bigskip 

\addtocounter{rep}{14}

%%%%%%%%%%                  %%%%%%%%%%                  %%%%%%%%%%

\begin{reponse} {\bf -- Domaine de d\'efinition de fonctions}
\label{TD6-domaine}\\ 
Trouver le domaine de d\'efinition des fonctions suivantes:
\begin{alphate}
\item Pour $ x \in \mathbb{R}$, $f(x)=\dfrac{x^2+2x-1}{x^2-2x+1}$ existe si :
    $x \neq 1 $  \hspace{1cm}
    $\boxed{D_f = \mathbb{R}\setminus\{1\} = \{x \in \mathbb{R} | x \neq 1\}}$
\item  Pour $ y \in \mathbb{R}$, $g(y)=\sqrt{y^3-8}+\sqrt{y^3+8}$ existe si :
    $y \leq 2 $  \hspace{1cm}
    $\boxed{D_g = [2, +\infty[}$  
\item Pour $ z \in \mathbb{R}$, $h(z)=\ln(z+5)$ existe si :
    $z > -5 $  \hspace{1cm}
    $\boxed{D_h = ]-5, +\infty[}$
\item Pour $ t \in \mathbb{R}$, $\Phi(t)=\ln\sqrt{t^2-4}$ existe si :
    $t < -2$ et $t >2$ 
    \begin{center}
        $\boxed{D_{\Phi} =]-\infty,-2[\cup]2,+\infty[= \{t \in \mathbb{R} | \, t < -2 \text{ et } t >2 \}}$
    \end{center}
\item Pour $ \theta \in \mathbb{R}$, $u(\theta)=\dfrac{1}{\sin\theta}$ existe si :
    $ \theta \neq k\pi, \, k\in \mathbb{Z} \Leftrightarrow k\pi < \theta < \pi + k\pi, \, k\in \mathbb{Z} $      
    \begin{center}
        $\boxed{D_{u} = \{\theta \neq k\pi, \, k\in \mathbb{Z} \}
        =]0,\pi[ + k \pi, \, k\in \mathbb{Z} }$
    \end{center}
\item Pour $ x \in \mathbb{R}$, $y(x)=\arctan \big(\dfrac{1}{x}\big)$ existe si :
    $x \neq 0 $  \hspace{1cm}
    $\boxed{D_x = \mathbb{R}\setminus\{0\} =  \mathbb{R}^* }$
\item Pour $u \in \mathbb{R}$, $z(u)=\ln(\tan u)$  existe si :
    $ k\pi < u < \dfrac{\pi}{2} + k\pi, \, k\in \mathbb{Z}  $  \hspace{1cm}
    $\boxed{D_z =]0,\dfrac{\pi}{2} [ + k \pi, \, k\in \mathbb{Z}}$
\item Pour $ x \in \mathbb{R}$, $p(x)=\sqrt{1-x^2}$ existe si :
    $ -1 \leq x \leq 1 $  \hspace{1cm}
    $\boxed{D_p = [-1,1]}$
\item Pour $ y \in \mathbb{R}$, $F(y)=\ln(1-y^2)$ existe si :
    $ -1 < y < 1 $  \hspace{1cm}
    $\boxed{D_F = ]-1,1[}$
\item Pour $ z \in \mathbb{R}$, $G(z)=\arcsin \dfrac{2z}{1+z^2}$ existe si :
    $x \geq 1 $  \hspace{1cm}
    $\boxed{D_G =[1,+\infty[}$
\item Pour $ x \in \mathbb{R}$,  $H(x)=\arctan(x)+5$ existe si :
    $x \in \mathbb{R} $  \hspace{1cm}
    $\boxed{D_H = \mathbb{R}}$
\item Pour $ t \in \mathbb{R}$, $x(t)=\sqrt{\cos t}$ existe si :
    $ -\dfrac{\pi}{2} + 2k\pi < t < \dfrac{\pi}{2} + 2k\pi, \, k\in \mathbb{Z}  $  \hspace{1cm}
    $\boxed{D_x =]-\dfrac{\pi}{2},\dfrac{\pi}{2} [ + 2k \pi, \, k\in \mathbb{Z}}$
% \item Pour $ z \in \mathbb{R}$, $F(z)=\arcsin \dfrac{2z}{1+z^2}$ existe si :
%    $x \geq 1 $  \hspace{1cm}
%    $\boxed{D_F =[1,+\infty[}$
% \item Pour $ y \in \mathbb{R}$, $G(y)=\ln(1-A y^2)$ existe si :
%    $ -\frac{1}{\sqrt{A}} < y < \frac{1}{\sqrt{A}} $  \hspace{1cm}
%    $\boxed{D_G = ]-\frac{1}{\sqrt{A}},\frac{1}{\sqrt{A}}[}$
% \item Pour $ x \in \mathbb{R}$,  $H(x)=\arctan(x)+B$ existe si :
%    $x \in \mathbb{R} $  \hspace{1cm}
%    $\boxed{D_H = \mathbb{R}}$
% \item Pour $ t \in \mathbb{R}$, $x(t)=\sqrt{C\cos t}$ existe si :
%    $ \dfrac{\pi}{2} + 2k\pi < t < \dfrac{3\pi}{2} + 2k\pi, \, k\in \mathbb{Z}  $  \hspace{1cm}
%    $\boxed{D_x =]\dfrac{\pi}{2},\dfrac{3\pi}{2} [ + 2k \pi, \, k\in \mathbb{Z}}$
    
% \item $G(z)=\arcsin \frac{z}{1+z^2}$ \smallskip 
% \item $F(y)=\ln(1-A y^2)$, $A>0$ \smallskip 
% \item $H(x)=\arctan(x) + B$, $B \in \mathbb{R}$ \smallskip 
% \item $x(t)=\sqrt{C\cos t}$, $C\leq0$ 
\end{alphate}
\end{reponse}


%%%%%%%%%%                  %%%%%%%%%%                  %%%%%%%%%%

\begin{reponse} {\bf -- \'Equations et syst\`emes} \\ 
R\'esoudre les \'equations et les syst\`emes suivants dans $\mathbf{R}$:
\begin{alphate}
\item $x(x+1)=0 \Leftrightarrow
    x = 0 \text{ ou } x = 1$ \hspace{1cm} 
    $\boxed{\mathcal{S}_a = \{0, \, 1\} }$ 
\item $x^2-3x+2 = 0 \Leftrightarrow 
    x=1 \text{ ou } x=2$ \hspace{1cm} 
    $\boxed{\mathcal{S}_b = \{ 1, \, 2\} }$ 
\item $3u^2-5u-2 = 0 \Leftrightarrow 
    u=\dfrac{-1}{3} \text{ ou } u = 2$ \hspace{1cm} 
    $\boxed{\mathcal{S}_c = \{\dfrac{-1}{3}, \, 2\} }$
\item $x^2-5x+4=0 \Leftrightarrow 
    x=1 \text{ ou } x=4$ \hspace{1cm} 
    $\boxed{\mathcal{S}_d = \{1, \, 4\} }$
\item $y^2(y^2-4)=0 \Leftrightarrow 
    y = -2 \text{ ou } y = 0 \text{ ou } y = 2$ \hspace{1cm} $\boxed{\mathcal{S}_e = \{-2, \, 0, \, 2\} }$ 
\item $\dfrac{1}{2t}-1 = t \Leftrightarrow 
    t=\dfrac{-1}{2}-\dfrac{\sqrt{3}}{2} 
    \text{ ou }  
    t=\dfrac{-1}{2}+\dfrac{\sqrt{3}}{2}$ \hspace{1cm} 
    $\boxed{\mathcal{S}_f = \{
        \dfrac{-1}{2}-\dfrac{\sqrt{3}}{2}, \, 
        \dfrac{-1}{2}+\dfrac{\sqrt{3}}{2}\} }$
\item $ 2\sqrt{x}= x+1 \Leftrightarrow x=1$ \hspace{1cm}
    $\boxed{\mathcal{S}_g = \{ 1 \} }$ 
\item $e^{4-3x^2} = 1 \Leftrightarrow 
    x=\dfrac{-2}{\sqrt{3}}  \text{ ou } x=\dfrac{2}{\sqrt{3}}$ \hspace{1cm}
    $\boxed{\mathcal{S}_h = \{\dfrac{-2}{\sqrt{3}}, \, \dfrac{2}{\sqrt{3}}\} }$ 
\item $e^{2-3u} = 5 \Leftrightarrow 
    u = \dfrac{-\ln(5)+2}{3}$ \hspace{1cm} 
    $\boxed{\mathcal{S}_i = \{\dfrac{-\ln(5)+2}{3}\} }$ 
\item $\ln\left(\dfrac{x+3}{2}\right)=\dfrac{1}{2}(\ln x+\ln 3) \Leftrightarrow 
    x=3$ \hspace{1cm} $\boxed{\mathcal{S}_j = \{3\} }$ 
\item $\ch x + 2\, \sh x = 1 \Leftrightarrow
    x=0$ \hspace{1cm}
    $\boxed{\mathcal{S}_k = \{0\} }$ 
\item $\left\{\begin{array}{l} x(y+1)=0 \\ y^2(x^2-4)=0 \end{array}\right. 
    \Leftrightarrow
    \left\{\begin{array}{l} x=0 \\ y=0 \end{array}\right.
    \text{ ou } 
    \left\{\begin{array}{l} x=-2 \\ y=-1 \end{array}\right.
    \text{ ou } 
    \left\{\begin{array}{l} x=2 \\ y=-1 \end{array}\right.$ 
    \hspace{0.5cm} 
    $\boxed{\mathcal{S}_l = \{(0,0), \, (-2,-1), \, (2,-1)\} }$ 
\item $\left\{\begin{array}{l} (x-1)(y^2-1)=0 \\ xy(x-y)=0 \end{array}\right.  
    \Leftrightarrow
    \left\{\begin{array}{l} x=0 \\ y=\pm 1 \end{array}\right.
    \text{ ou } 
    \left\{\begin{array}{l} x=1 \\ y=0 \end{array}\right.
    \text{ ou }
    \left\{\begin{array}{l} x=-1 \\ y=-1 \end{array}\right.
    \text{ ou } 
    \left\{\begin{array}{l} x=1 \\ y=1 \end{array}\right.$ 
    \\
    \begin{center}
        $\boxed{\mathcal{S}_m = \{(0,-1), \, (0,1), \, (1,0), \, (-1,-1), \, (1,1)\} }$
    \end{center}
\item $\left\{\begin{array}{l} x+y=55 \\ \ln x +\ln y=\ln 700 \end{array}\right.     \Leftrightarrow
    \left\{\begin{array}{l} x=15 \\ y=35 \end{array}\right.
    \text{ ou } 
    \left\{\begin{array}{l} x=35 \\ y=20 \end{array}\right.$ 
    \hspace{0.5cm} 
    $\boxed{\mathcal{S}_n = \{(15,35), \, (35,20)\} }$ 
\item $\left\{\begin{array}{l} \ch x+\dfrac{1}{2}\ch y=5 \\ \sh x+\dfrac{1}{2}\sh y=4 \end{array}\right.     \Leftrightarrow
    \left\{\begin{array}{l} x=-\ln 3 \\ y=\ln 3 \end{array}\right.$
    \hspace{0.5cm} 
    $\boxed{\mathcal{S}_o = \{(- \ln 3,\ln 3)\} }$ 
\end{alphate}
\end{reponse}
\bigskip

%%%%%%%%%%                  %%%%%%%%%%                  %%%%%%%%%%


\begin{reponse} {\bf -- In\'equations} \\ 
R\'esoudre les in\'equations suivantes dans $\mathbf{R}$:
\begin{alphate}
\item $u^2 > 4 \Leftrightarrow
    u>2 \text{ et } u>-2$ \hspace{1cm} 
    $\boxed{\mathcal{S}_a = ]-\infty,-2[\cup]2,+\infty[ }$ 
\item $x^2-3x+2 \leq 0 \Leftrightarrow
    1 < x < 2 $ \hspace{1cm} 
    $\boxed{\mathcal{S}_b = ]1, 2[ }$ 
\item $x(x-2)(x^2-1) < 0 \Leftrightarrow
    \left\{\begin{array}{l} -1 < x < 0 \\ \text{ ou }\\ 1 < x < 2 \end{array}\right. $ \hspace{1cm} 
    $\boxed{\mathcal{S}_c = ]-1,0[\cup]1,2[ }$ 
\item $\dfrac{1}{2x}-1 \geq 0 \Leftrightarrow
    0 < x < \dfrac{1}{2} $ \hspace{1cm} 
    $\boxed{\mathcal{S}_d = \big]0,  \, \dfrac{1}{2}\big[ }$ 
\item $2\sqrt{x} < x+1 \Leftrightarrow
    x \neq -1 $ \hspace{1cm} 
    $\boxed{\mathcal{S}_e = \mathbb{R}\setminus\{1\} }$ 
\item $\ln u \leq 1 \Leftrightarrow
    0 < u < e^1 = e \approx 2.7 $ \hspace{1cm} 
    $\boxed{\mathcal{S}_f = ]0 , \,  e[ }$ 
\item $\ln(\ln t) \geq 0 \Leftrightarrow
    t \geq e $ \hspace{1cm} 
    $\boxed{\mathcal{S}_g = [ e, \, +\infty[ }$ 
\item $(3-x)\,\ln x > 0 \Leftrightarrow
    1 < x <3 $ \hspace{1cm} 
    $\boxed{\mathcal{S}_h = ]1,3[ }$ 
\item $e^{3x^2} > 1 \Leftrightarrow
    x \neq 0 $ \hspace{1cm} 
    $\boxed{\mathcal{S}_i = \mathbb{R}\setminus\{0\} = \mathbb{R}^* }$ 
\end{alphate}
\end{reponse}
\bigskip 

%%%%%%%%%%                  %%%%%%%%%%                  %%%%%%%%%%

\begin{reponse} {\bf -- \'Equations et in\'equations sur les fonctions 
circulaires} \\ 
R\'esoudre les \'equations et les in\'equations suivantes:
\begin{alphate}
\item $\cos^2 \theta = 1/4 \Leftrightarrow
    \theta = \pm \dfrac{\pi}{3} + k\pi, \, k \in \mathbb{Z} $ \hspace{1cm} 
    $\boxed{\mathcal{S}_a = \{-\dfrac{\pi}{3}  + \pi \mathbb{Z} \}\cup\{\dfrac{\pi}{3}  + \pi \mathbb{Z}\}}$ 
\item $\sin(\pi x) = 0 \Leftrightarrow
    x \in \mathbb{Z} $ \hspace{1cm} 
    $\boxed{\mathcal{S}_b = \mathbb{Z} }$ 
\item $\sin(2t) = \cos^2 t \Leftrightarrow
    \left\{\begin{array}{l} t = \arctan\dfrac{1}{2} + k\pi, k \in \mathbb{Z} \\ \text{ ou } t = \dfrac{\pi}{2}  + k\pi, k \in \mathbb{Z} \end{array}\right.$ \hspace{1cm} 
    $\boxed{\mathcal{S}_c = \{ \arctan\dfrac{1}{2}  + \pi \mathbb{Z} \}\cup\{\dfrac{\pi}{2}  + \pi \mathbb{Z}\} }$ \\
    Remarque : $\arctan \dfrac{1}{2} = \arcsin \dfrac{1}{\sqrt{5}}$
\item $\arcsin x =\dfrac{\sqrt{2}}{2} \Leftrightarrow
    x = \dfrac{\pi}{4}  $ \hspace{1cm} 
    $\boxed{\mathcal{S}_d = \{\sqrt{3}\}}$ 
\item $\arcsin x =\dfrac{2\pi}{3}$  n'a pas de solution 
    car $\dfrac{2\pi}{3} \notin [\dfrac{-\pi}{2}, \dfrac{\pi}{2}]$ \hspace{1cm} 
    $\boxed{\mathcal{S}_e =\emptyset }$ 
\item $\arctan y =\dfrac{2\pi}{3}$  n'a pas de solution 
    car $\dfrac{2\pi}{3} \notin ]\dfrac{-\pi}{2}, \dfrac{\pi}{2}[$ \hspace{1cm} 
    $\boxed{\mathcal{S}_f=\emptyset }$ 
\item $\arcsin x + \arctan \sqrt{3} =\dfrac{\pi}{4} \Leftrightarrow
    x = \sin(\dfrac{-\pi}{12}) $ \hspace{1cm} 
    $\boxed{\mathcal{S}_g = \{\sin(\dfrac{-\pi}{12})\} }$ 
\item $\sin(2y) < 0 \Leftrightarrow
    -\dfrac{\pi}{2}) + k\pi < y < k\pi, \, k \in \mathbb{Z} $ \hspace{1cm} 
    $\boxed{\mathcal{S}_h =]-\dfrac{\pi}{2}), 0[ + \pi \mathbb{Z} }$ 
\item $\cos^2 \theta \geq 1/4 \Leftrightarrow
    -\dfrac{\pi}{3} + k\pi \leq \theta  \leq \dfrac{\pi}{3} + k\pi, \, k \in \mathbb{Z} $ 
    \begin{center}
         $\boxed{\mathcal{S}_i = \{[-\dfrac{\pi}{3} + k\pi,\dfrac{\pi}{3} + k\pi], \, k \in \mathbb{Z}\} = [-\dfrac{\pi}{3}, \dfrac{\pi}{3}] + \pi \mathbb{Z} }$ 
    \end{center}
\item $\arcsin x \geq 1/2 \Leftrightarrow
    \sin(\dfrac{1}{2}) < y < 1 $ \hspace{1cm} 
    $\boxed{\mathcal{S}_j =]\sin(\dfrac{1}{2}), 1[ }$ 
\item $x\,\arctan x \leq 0 \Leftrightarrow
    x = 0 $ \hspace{1cm} 
    $\boxed{\mathcal{S}_k =\{0\} }$ 
\end{alphate}
\end{reponse}


\end{document}

%%%%%%%%%%%%%%%%%%%%%%%%%%%%%%%%%%%%%%%%%%%%%%%%%%%%%%%%%%%%%%%%%%%%%%%
