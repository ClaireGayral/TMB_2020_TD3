%
%   TMB-TD-2019.tex  - Tine Léon Matar
%
%%%%%%%%%%%%%%%%%%%%%%%%%%%%%%%%%%%%%%%%%%%%%%%%%%%%%%%%%%%%%%%%%%%%%%

\documentclass[a4paper,11pt]{article}

\usepackage{amsfonts,amsmath,amssymb,amscd,amstext,mathabx}
\usepackage{textcomp,multicol,enumitem,mdwlist}
\usepackage{pict2e,graphicx}

\usepackage[T1]{fontenc}
\usepackage[french]{babel}

\setlength{\textheight}{27cm}
\setlength{\textwidth}{18cm}
\setlength{\topmargin}{-2.8cm}
%\setlength{\leftmargin}{-.5cm}
\setlength{\oddsidemargin}{-1.05cm}
\setlength{\evensidemargin}{-1.05cm}
 
%\setcounter{page}{0}

\renewcommand{\arraystretch}{1.5}

\newcommand{\bitem}{\item[$\bullet$]}
\newcommand{\ds}{\displaystyle}

\newcommand{\R}{\mathbb R}
\newcommand{\C}{\mathbb C}
\newcommand{\N}{\mathbb N}
\newcommand{\Z}{\mathbb Z}
\newcommand{\Q}{\mathbb Q}
\newcommand{\dist}{\mathrm{dist}\,}
\newcommand{\Vect}{\mathrm{Vect}\,} 
\newcommand{\Lin}{\mathrm{Lin}\,} 
\newcommand{\calF}{{\cal F}}
\newcommand{\calM}{{\cal M}}
\newcommand{\Aire}{\mathrm{Aire}\,}
\newcommand{\ch}{\mathrm{ch}\,}
\newcommand{\sh}{\mathrm{sh}\,}
\renewcommand{\th}{\mathrm{th}\,}

\newcommand{\vi}{\vec{\mbox{\it \i}}\,}
\newcommand{\vj}{\vec{\mbox{\it \j}}\,} 
\newcommand{\vk}{\vec{\mbox{\it k}}\,} 
\newcommand{\ve}{\vec{e}}
\newcommand{\vu}{\vec{u}}
\newcommand{\vv}{\vec{v}}
\newcommand{\vw}{\vec{w}}
\newcommand{\vh}{\vec{h}}
\newcommand{\vx}{\vec{x}}
\newcommand{\vy}{\vec{y}}
\newcommand{\vXY}[2]{\left(\!\!\begin{array}{c} #1 \\ #2 \end{array}\!\!\right)}
\newcommand{\vXYZ}[3]{\left(\!\!\begin{array}{c} #1 \\ #2 \\ #3 \end{array}\!\!\right)}

\newcommand{\vA}{\overrightarrow{A}}
\newcommand{\vB}{\overrightarrow{B}}
\newcommand{\vE}{\overrightarrow{E}}
\newcommand{\vF}{\overrightarrow{F}}
\newcommand{\vV}{\overrightarrow{V}}
\newcommand{\vU}{\overrightarrow{U}}

\newcommand{\OXY}{\begin{picture}(30,20)(0,-5)
                  \put(-20,0){\vector(1,0){45}}
                  \put(17,-7){\footnotesize $x$}
                  \put(0,-20){\vector(0,1){45}}
                  \put(-7,17){\footnotesize $y$}
                  \end{picture}}
\newcommand{\OXYZ}{\begin{picture}(30,10)(0,0)
                   \put(10,20){\vector(-1,-2){20}}
                   \put(-16,-18){\footnotesize $x$}
                   \put(-20,0){\vector(1,0){45}}
                   \put(19,-7){\footnotesize $y$}
                   \put(0,-20){\vector(0,1){45}}
                   \put(-7,19){\footnotesize $z$}
                   \end{picture}}


\newenvironment{alphate}{\begin{enumerate}[label=\alph*)]}{\end{enumerate}}

\newtheorem{exo}{Exercice}
\newenvironment{exercice}{\begin{exo} \em}{\end{exo}}

\newtheorem{rep}{R\'eponses}
\newenvironment{reponse}{\begin{rep} \em}{\end{rep}}


%%%%%%%%%%%%%%%%%%%%%%%%%%%%%%%%%%%%%%%%%%%%%%%%%%%%%%%%%%%%%%%%%%%%%%

\begin{document}

%%%%%%%%%%%%%%%%%%%%%%%%%%%%%%%%%%%%%%%%%%%%%%%%%%%%%%%%%%%%%%%%%%%%%%%



\begin{center}
{\bf TD 3 \quad--\quad \'EQUATIONS, IN\'EQUATIONS ET DOMAINE DE 
D\'EFINITION}
\end{center}
\bigskip 

\addtocounter{exo}{14}

%%%%%%%%%%                  %%%%%%%%%%                  %%%%%%%%%%

\begin{exercice} {\bf -- Domaine de d\'efinition de fonctions}
\label{TD6-domaine}\\ 
Trouver le domaine de d\'efinition des fonctions suivantes:
\begin{multicols}{3}
\begin{alphate}
\item $f(x)=\dfrac{x^2+2x-1}{x^2-2x+1}$ \smallskip 
\item $g(y)=\sqrt{y^3-8}+\sqrt{y^3+8}$ \smallskip 
\item $h(z)=\ln(z+5)$ \smallskip 
\item $\Phi(t)=\ln\sqrt{t^2-4}$
\item $u(\theta)=\dfrac{1}{\sin\theta}$ \smallskip 
\item $y(x)=\arctan \big(\frac{1}{x}\big)$ \smallskip 
\item $z(u)=\ln(\tan u)$ \smallskip 
\item $p(x)=\sqrt{1-x^2}$
\item $F(y)=\ln(1-y^2)$ \smallskip 
\item $G(z)=\arcsin \frac{2z}{1+z^2}$ \smallskip 
\item $H(x)=\arctan(x)+5$ \smallskip 
\item $x(t)=\sqrt{\cos t}$ 
\end{alphate}
\end{multicols}
\end{exercice}
\bigskip

%%%%%%%%%%                  %%%%%%%%%%                  %%%%%%%%%%

\begin{exercice} {\bf -- \'Equations et syst\`emes} \\ 
R\'esoudre les \'equations et les syst\`emes suivants:
\begin{multicols}{3}
\begin{alphate}
\item $x(x+1)=0$ \smallskip 
\item $x^2-3x+2 = 0$ \smallskip 
\item $3u^2-5u-2 = 0$ \smallskip 
\item $x^2-5x+4=0$ \smallskip  
\item $y^2(y^2-4)=0$ \smallskip 
\item $\dfrac{1}{2t}-1 = t$ \smallskip 
\item $ 2\sqrt{x}= x+1 $
\item $e^{4-3x^2} = 1$ \smallskip 
\item $e^{2-3u} = 5$ \smallskip 
\item $\ln\left(\dfrac{x+3}{2}\right)=\dfrac{1}{2}(\ln x+\ln 3)$ \smallskip 
\item $\ch x + 2\, \sh x = 1$ \smallskip
\item $\left\{\begin{array}{l} x(y+1)=0 \\ y^2(x^2-4)=0 \end{array}\right.$ \smallskip 
\item $\left\{\begin{array}{l} (x-1)(y^2-1)=0 \\ xy(x-y)=0 \end{array}\right.$ \smallskip 
\item $\left\{\begin{array}{l} x+y=55 \\ \ln x +\ln y=\ln 700 \end{array}\right.$ \smallskip 
\item $\left\{\begin{array}{l} \ch x+\frac{1}{2}\ch y=5 \\ \sh x+\frac{1}{2}\sh y=4 \end{array}\right.$ \smallskip 
\end{alphate}
\end{multicols}
\end{exercice}
\bigskip

\begin{exercice} {\bf -- In\'equations} \\ 
R\'esoudre les in\'equations suivantes:
\begin{multicols}{3}
\begin{alphate}
\item $u^2 > 4$ \smallskip 
\item $x^2-3x+2 \leq 0$ \smallskip 
\item $x(x-2)(x^2-1) < 0$ \smallskip 
\item $\dfrac{1}{2x}-1 \geq 0$ \smallskip 
\item $2\sqrt{x} < x+1$ \smallskip 
\item $\ln u \leq 1$ \smallskip 
\item $\ln(\ln t) \geq 0$ \smallskip 
\item $(3-x)\,\ln x > 0$ \smallskip 
\item $e^{3x^2} > 1$ \smallskip 
\end{alphate}
\end{multicols}
\end{exercice}
\bigskip 

%%%%%%%%%%                  %%%%%%%%%%                  %%%%%%%%%%

\begin{exercice} {\bf -- \'Equations et in\'equations sur les fonctions 
circulaires} \\ 
R\'esoudre les \'equations et les in\'equations suivantes:
\begin{multicols}{3}
\begin{alphate}
\item $\cos^2 \theta = 1/4$ \smallskip 
\item $\sin(\pi x) = 0$ \smallskip 
\item $\sin(2t) = \cos^2 t$ \smallskip 
\item $\arcsin x =\dfrac{-\pi}{3}$ \smallskip 
\item $\arcsin x =\dfrac{2\pi}{3}$ \smallskip 
\item $\arctan y =\dfrac{2\pi}{3}$ \smallskip 
\item $\arcsin x + \arctan \sqrt{3} =\dfrac{\pi}{4}$ \smallskip 
\item $\sin(2y) < 0$
\item $\cos^2 \theta \geq 1/4$
\item $\arcsin x \geq 1/2$
\item $x\,\arctan x \leq 0$
\end{alphate}
\end{multicols}
\end{exercice}
\bigskip 

%%%%%%%%%%                  %%%%%%%%%%                  %%%%%%%%%%

%\begin{exercice}{\bf -- Pour aller plus loin} \\ 
%R\'esoudre les \'equations et les in\'equations suivantes:
%\begin{alphate}
%\item $\cos 4x = \sin 7x$ \smallskip 
%\item $\ch x + 2\, \sh x \leq 1$ \smallskip 
%\end{alphate}
%\end{exercice}

\end{document}

%%%%%%%%%%%%%%%%%%%%%%%%%%%%%%%%%%%%%%%%%%%%%%%%%%%%%%%%%%%%%%%%%%%%%%%
