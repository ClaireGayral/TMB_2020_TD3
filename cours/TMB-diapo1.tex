%
%   TMB-diapo1.tex  2019-2020  A. Mikelic & L. M. Tine
%
%%%%%%%%%%%%%%%%%%%%%%%%%%%%%%%%%%%%%%%%%%%%%%%%%%%%%%%%%%%%%%%%%%%%%%

\section{1 Complexes}

\begin{frame}%%[plain]
%\only<1>{\addtocounter{framenumber}{-1}}  %\addtocounter{framenumber}{-1}
\frametitle{\bf TMB -- Chapitre 1 \\ 
Nombres complexes}

Dans ce chapitre: 
\begin{itemize}
\item[1.] 
Nombres complexes (coordonnées, opérations élémentaires). 

\item[2.] 
Changements de coordonnées. 

\item[3.] 
Polyn\^omes complexes (racines, factorisation). 
\end{itemize}

\end{frame}

%%%%%%%%%%%%%%%%%%%%%%%%%%%%%%%%%%%%%%%%%%%%%%%%%%
%%%%%%%%%%%%%%%%%%%%%%%%%%%%%%%%%%%%%%%%%%%%%%%%%%

%\addfootboxtemplate{\color{white}}{\color{black}\hfill\large{\insertframenumber}\hspace{1em}\null}
\begin{frame}
\begin{center}
1. Nombres complexes (coordonnées, opérations élémentaires). 
\end{center}
\end{frame}

\begin{frame}
\frametitle{\bf Nombres complexes}
\medskip 

{\bf D\'efinition:} 
Un {\bf nombre complexe} $\mathbf z$ est un point du plan cart\'esien. 
\begin{center}
\begin{picture}(100,40)(-50,-10)
\put(-50,0){\vector(1,0){100}}
\put(0,-10){\vector(0,1){40}}
\put(25,10){\circle*{2}}
\put(20,14){\scriptsize $z$}
\put(25,0){\circle*{2}}
\multiput(0,10)(4,0){6}{\line(1,0){2}}
\put(0,10){\circle*{2}}
\put(-6,10){}
\multiput(25,0)(0,4){3}{\line(0,1){2}}
\end{picture}
\end{center}
{\small $\C$ est l'{ensemble des nombres complexes} et on écrit 
$z\in\C$ pour exprimer le fait que $z$ est un nombre complexe. }

\pause

\vspace*{1mm} 

{\bf Exemples:}
\begin{itemize}
\item[$\bullet$]
\framebox{${z= i}$ est le point dont l'abscisse vaut $x=0$ et l'ordonnée $y=1$.}
\item[$\bullet$]
Les {\bf nombres r\'eels} forment l'axe des abscisses. On les note $z=x, \text{ où $x\in\R$}$.
\item[$\bullet$]
Les {\bf nombres imaginaires} forment l'axe des ordonnées. Ce sont les multiples de $i$. On les note $z=iy, \text{ où $y\in\R$}$
\item[$\bullet$]
Tout nombre complexe s'écrit \framebox{${z=x+iy}$}, où $x\in\R$ et $y\in\R$ 
\vspace*{1mm}
\end{itemize}
\begin{center}
\begin{picture}(100,40)(-50,-10)
%\thicklines
\put(-50,0){\vector(1,0){100}}
\put(0,-10){\vector(0,1){40}}
\put(25,10){\circle*{2}}
\put(20,14){\scriptsize $z=x+iy$}
\put(25,0){\circle*{2}}
\put(25,-6){\scriptsize $x$}
\multiput(0,10)(4,0){6}{\line(1,0){2}}
\put(0,10){\circle*{2}}
\put(-6,10){\scriptsize $iy$}
\put(0,20){\circle*{2}}
\put(-6,20){\scriptsize $i$}
\multiput(25,0)(0,4){3}{\line(0,1){2}}
\end{picture}
\end{center}
{\small \it NB: sur ce dessin, $x\simeq1,1$ et $y\simeq0,5$.}
\end{frame}
%%%%%%%%%%%%%%%%%%%%%%%%%%%%%%%%%%%%%%%%%%%

\begin{frame}%[plain]
%\only<1>{\addtocounter{framenumber}{-1}}
\frametitle{\bf Coordonnées cartésiennes}
\medskip 

{\bf D\'efinition:} 
Pour un nombre complexe $z=x+iy$, où $x\in\R$ et $y\in\R$, 
\begin{itemize}
\bitem
$x$ est la {\bf partie r\'eelle} de $z$ et on écrit \framebox{\ $x=\Re(z)$} 
\bitem
$y$ est la {\bf partie imaginaire} de $z$ et on écrit \framebox{\ $y=\Im(z)$} 
%\bitem
%{\bf conjugu\'e complexe} le nombre complexe\
%\framebox{\ $\overline{z}=x-iy$\ } 
\bitem
$(x,y)$ sont appelées les {\bf coordonnées cartésiennes} de $z$.
\end{itemize}
\vspace*{3mm}

\pause
{\small 
{\bf Exemple : } Pour $z=3-2i$, on a $\Re(z)=3$, $\Im(z)=-2$ et $(3,-2)$ sont les coordonnées cartésiennes de $z$.}
\vspace*{4mm}

{\bf Application: }
le {\bf nombre complexe conjugué} de $z$ est défini par \framebox{\ $\overline{z}=x-iy$\ }.

$\overline z$ est donc le point symétrique de $z$ par rapport à l'axe des abscisses.
\vspace*{1mm}
\begin{center}
\begin{picture}(100,40)(-50,-10)
%\thicklines
\put(-50,0){\vector(1,0){100}}
\put(0,-10){\vector(0,1){40}}
\put(25,10){\circle*{2}}
\put(20,14){\scriptsize $z=x+iy$}
\put(25,-10){\circle*{2}}
\put(20,-14){\scriptsize $\overline{z}=x-iy$}
%\multiput(0,10)(4,0){6}{\line(1,0){2}}
\multiput(25,-10)(0,7){3}{\line(0,1){2}}
\end{picture}
\end{center}
{\small 
{\bf Exemple : } Pour $z=2+i$, on a $\overline{z}=2-i$. Que vaut le conjugué de ${\overline z}$ ?}
\end{frame}

%%%%%%%%%%%%%%%%%%%%%%%%%%%%%%%%%%%%%%%%%%%

\begin{frame}%[plain]
%\only<1>{\addtocounter{framenumber}{-1}}
\frametitle{\bf Coordonnées polaires}
\medskip 

Outre les coordonnées cartésiennes, on peut identifier un nombre complexe $z$ comme suit.

\medskip

{\bf D\'efinition:} 
\begin{itemize}
\bitem
Le {\bf module} de $z$ est la distance de $z$ à l'origine, notée \framebox{${\rho=\left|z\right|}$}. %\vspace*{1mm}
\bitem
{\bf Un argument} de $z$ est un angle\footnote{orienté dans le sens trigonométrique} $\theta$ formé par la demi-droite $\R_+$ et la demi-droite $[Oz)$.
\begin{center}
\begin{tikzpicture}[scale=0.5]
    %% axis %%
    \draw[->,>=stealth'] (-1,0) -> (3.5,0)  node[right]{\scriptsize $\R_+$};
    \draw[->,>=stealth'] (0,-1) -> (0,2.5) ;
    \draw (0,0) circle (0.05)[fill=black] node[below left] {\scriptsize $O$};

    %% dot  z %% :
    \draw (30:3) circle(0.05)[fill=black] node[above]{\scriptsize $z=\rho e^{i\theta}$} ;
    \draw (0,0) -> (30:3) node[midway, above]{\scriptsize $\rho$};
    
    %% theta arc %%

    \draw[->,densely dotted] (0:1) arc[radius=1, start angle=0, end angle=30];
    \draw (20:1) node[right] {\scriptsize $\theta$} ;
\end{tikzpicture}
\end{center}
%\begin{center}
%\begin{picture}(80,50)(-20,-10)
%%\thicklines
%\put(-20,0){\vector(1,0){80}}
%\put(0,-10){\vector(0,1){50}}
%\put(0,0){\circle*{2}}
%\put(-6,-6){\scriptsize $O$}
%\put(40,20){\circle*{2}}
%\put(63,-2){\scriptsize $\R_+$}
%\put(30,24){\scriptsize $z=\rho e^{i\theta}$}
%\put(0,0){\line(2,1){40}}
%\put(18,13){\scriptsize $\rho$}
%\multiput(20,0)(-1,2){4}{\line(-2,1){1}}
%\put(20,3){\scriptsize $\theta$}
%\end{picture}
%\end{center}
\bitem 
{\bf L'argument principal} de $z$ est l'unique argument de $z$ appartenant à $]-\pi,\pi]$. 

On note \framebox{${\theta=\arg(z)\in]-\pi,\pi]}$}.
\bitem
$(\rho,\theta)$ sont appelées les {\bf coordonnées polaires} de $z$ et on note \framebox{$z=\rho e^{i\theta}$}.
\end{itemize}

{\small 
{\bf Exemple : } 
pour $z=i$, $\vert i\vert=1$, $\arg(i)=\pi/2$  et $\theta=5\pi/2$ est un autre argument de $i$. Donc, \framebox{$i= e^{i\frac\pi2}$}$=e^{i\frac{5\pi}2}$
%\begin{center}
%\begin{picture}(80,30)(-20,-10)
%%\thicklines
%\put(-20,0){\vector(1,0){80}}
%\put(0,-10){\vector(0,1){50}}
%\put(0,0){\circle*{2}}
%\put(-6,-6){\scriptsize $O$}
%\put(0,10){\circle*{2}}
%\put(-5,10){\scriptsize $i$}
%\multiput(10,0)(-2,2){4}{\line(-2,1){1}}
%\put(8,4){\scriptsize $\theta=\pi/2$ ou $5\pi/2$ ou ...}
%\put(-25,5){\scriptsize $\rho=1$}
%\end{picture}
%\end{center}
\medskip

{\bf Exercice : } montrer que $\boxed{-1=e^{i\pi}}$.
}

\end{frame}

%%%%%%%%%%%%%%%%%%%%%%%%%%%%%%%%%%%%%%%%%%%

\begin{frame}
\frametitle{\bf Addition et dilatation des nombres complexes}
\medskip 

{\bf D\'efinition:} 
On {\bf ajoute} deux nombres complexes à l'aide de leurs coordonnées cartésiennes: si $z=x+iy$ et $z'=x'+iy'$, alors
$$
\boxed{z+z' = (x+x')+i(y+y')}
$$
{\small On a de même $z-z' = (x-x')+i(y-y')$.}

{\bf Exemple: $z+\overline z = (x+iy) + (x-iy) = 2x$.}
\vspace*{2mm}

{\small {\bf Exercice (règle du parallélogramme):} représenter dans le plan cartésien $z=1+2i$, $z'=2+i$, $z''=z+z'$. Que pensez-vous du quadrilatère de sommets $O,z,z''$ et $z'$ ?}

\

{\bf D\'efinition:} 
On {\bf dilate} un nombre complexe en coordonnées cartésiennes: si $z=x+iy$ et $\lambda\in\R$, alors
$$\boxed{
\lambda\,z = (\lambda\,x)+i(\lambda\,y)
}
$$
{\small ou en coordonnées polaires {\it mais il faut faire attention au signe de $\lambda$}. Si $z=\rho e^{i\theta}$ et  
$$\boxed{
\text{si $\lambda\ge 0$, }\quad\lambda\,z = (\lambda\rho)e^{i\theta}
}
$$}
\vspace*{1mm}

{\small {\bf Exercice:} Soit $\lambda<0$ et $z\neq0$. L'expression $(\lambda\rho)e^{i\theta}$ n'est pas une écriture en coordonnées polaires car $\lambda\rho<0$ ne peut être une distance. Vérifier qu'en revanche $\lambda z = (\vert\lambda\vert\rho)e^{i(\theta+\pi)}$.}

\end{frame}

%%%%%%%%%%%%%%%%%%%%%%%%%%%%%%%%%%%%%%%%%%%

\begin{frame}
\frametitle{\bf Multiplication des nombres complexes}
\medskip 

{\bf D\'efinition:} 
On {\bf multiplie} deux nombres complexes à l'aide de leurs coordonnées polaires: si $z=\rho e^{i\theta}$ et $z'=\rho'e^{i\theta'}$, alors
$$
\boxed{zz' = \rho e^{i\theta}\cdot \rho'e^{i\theta'}=(\rho\rho')e^{i(\theta+\theta')}}
$$
{\small On a de même, quand $z'\neq0$, $\frac{z}{z'} = \frac{\rho e^{i\theta}}{\rho'e^{i\theta'}}=\frac{\rho}{\rho'}e^{i(\theta-\theta')}$.}
\medskip


{\bf Théorème : }$$\boxed{i^2=-1}$$

\medskip
{\small {\bf Démonstration: $i^2=i\cdot i = e^{i\pi/2}\cdot e^{i\pi/2}=e^{i\pi}=-1$.}}
\medskip

{\small {\bf Exercice:} vérifiez le théorème graphiquement en représentant $i=e^{i\pi/2}$ et $i^2=e^{i\pi}$.}
\medskip

{\bf Application (à retenir):} grâce au théorème, on peut aussi multiplier les nombres complexes en coordonnées cartésiennes:
\begin{align*}
z\cdot z' &= (x+iy)(x'+iy') \\
&= xx' + x(iy') + (iy)x' + (iy)(iy') \\
&= xx' + ixy' + ix'y - yy' \\
&= (xx' - yy') + i(xy'+x'y)
\end{align*}


\end{frame}

%%%%%%%%%%%%%%%%%%%%%%%%%%%%%%%%%%%%%%%%%%%


\begin{frame}%[plain]
%\only<1>{\addtocounter{framenumber}{-1}}
\frametitle{\bf Retour sur la conjugaison}
{\bf Rappel:} Si $z=x+iy$, où $x\in\R$ et $y\in\R$, alors $\boxed{\overline z=x-iy}$.
\medskip 

{\bf Propri\'et\'es (conjugaison et somme)} 
Soient $z$ et $z'$ deux nombres complexes. Alors,
\begin{itemize}
\bitem 
$\overline{z +z'}=\overline{z} +\overline{z'}$ 
\vspace*{2mm}
\bitem
$x=\boxed{\Re(z)=\frac{z+\overline{z}}{2}}$ \ et\quad   
$y=\boxed{\Im(z)=\frac{z-\overline{z}}{2i}}$
\vspace*{2mm}
\bitem 
{\ $z=\overline{z}$ si et seulement si $z$ est réel. }

\end{itemize}

{\bf Propri\'et\'es (conjugaison et produit)} 
\begin{itemize} 
\bitem 
$\overline{z z'}=\overline{z} \;\overline{z'}$ 

\bitem Si $z=\rho e^{i\theta}$, alors $\boxed{\overline z=\rho e^{-i\theta}}$

\bitem
$
\boxed{
\begin{aligned}
z\overline z &= (x+iy)(x-iy)=x^2+y^2\\
&=(\rho e^{i\theta})(\rho e^{-i\theta})=\rho^2
\end{aligned}
}
$
\end{itemize}

\medskip

{\small{\bf Application:} Exprimer $\frac{2}{2-i}$ en coordonnées cartésiennes.

{\it Idée:} On multiplie en haut et en bas par  $\overline{2-i}$ (pour appliquer la règle du $z\overline z$) :
$$
\frac{2}{2-i} = \frac{2(2+i)}{(2-i)(2+i)} = \frac{4+2i}{2^2+1^2} = \frac45+\frac 25i
$$
\medskip

{\bf Exercice:} 
\begin{itemize}
\item[-] vérifier graphiquement que ${\overline z=\rho e^{-i\theta}}$
\item[-] à l'aide des coordonnées polaires, vérifier que si $z'\neq0$, alors $\overline{z/z'}=\overline{z}/\overline{z'}$. 
\item[-] montrer que $z=-\overline{z}$ si et seulement si $z$ est imaginaire.
\end{itemize}}
\end{frame}

%%%%%%%%%%%%%%%%%%%%%%%%%%%%%%%%%%%%%%%%%%%

\begin{frame}
\frametitle{\bf Exemples: op\'erations en coordonnées cartésiennes}
\medskip 

{\bf Exemples:} 
Soient\quad $z=2\,(3+2i)$\quad et \quad $z'=5-i$. Alors:
$$
z+z' = (6+4i) + (5-i) = 11+3i 
\vspace*{-2mm}
$$

\pause
$$
z-z' = (6+4i) - (5-i) = 1+5i  
\vspace*{-2mm}
$$

\pause
\begin{align*}
zz' & = 2\,(3+2i)(5-i) \\ 
& = 2\,(15-3i+10i-2i^2) \\ 
& = 2\,\big((15+2) + (-3+10)i\big) \\ 
& = 2\,(17+7i) = 34+14i 
\vspace*{-2mm}
\end{align*}

\pause
\begin{align*}
\frac{z}{z'} & = \frac{2\,(3+2i)}{5-i} 
=  \frac{2\,(3+2i)(5+i)}{25+1} \\ 
& = \frac{2\,(15-2 + 10i+3i)}{26} \\ 
& = \frac{13}{13} + \frac{13}{13}i 
= 1 + i
\end{align*}

\end{frame}

%%%%%%%%%%%%%%%%%%%%%%%%%%%%%%%%%%%%%%%%%%%%%%%%%%

%%%%%%%%%%%%%%%%%%%%%%%%%%%%%%%%%%%%%%%%%%%

%\begin{frame}
%\frametitle{\bf Exemples: op\'erations en coordonnées polaires}
%\medskip 
%
%{\bf Exemple:} 
%Calculer $(z+\overline{z})(z^2+\overline{z}^2)$ en coordonnées polaires.
%
%\begin{align*}
%(z+\overline{z})(z^2+\overline{z}^2) &=z^3+ z\overline z^2+ \overline z z^2+ \overline{z}^3\\
%&=(\rho e^{i\theta})^3+(\rho e^{i\theta})(\rho e^{-i\theta})^2 + (\rho e^{-i\theta})(\rho e^{i\theta})^2+ (\rho e^{-i\theta})^3\\
%&=\rho^3e^{i3\theta}+\rho^3e^{-i\theta}+\rho^3e^{i\theta}+\rho^3e^{-i3\theta}\\
%&=
%\end{align*}
%
%
%\end{frame}



%%%%%%%%%%%%%%%%%%%%%%%%%%%%%%%%%%%%%%%%%%%%%%%%%%

\subsection{Rep. polaire} 

\begin{frame}
\begin{center}
2. Changements de coordonnées. 
\end{center}
\end{frame}


\begin{frame}%[plain]
%%\only<1>{\addtocounter{framenumber}{-1}}
\frametitle{\bf Des coordonnées polaires aux coordonnées cartésiennes}
\medskip 
\qquad
\begin{center}
\begin{picture}(80,50)(-20,-10)
%\thicklines
\put(-20,0){\vector(1,0){80}}
\put(0,-10){\vector(0,1){50}}
\put(40,20){\circle*{2}}
\put(0,0){\circle*{2}}
\put(-5,-5){\scriptsize $O$}
\put(36,24){\scriptsize $z=x+iy=\rho e^{i\theta}$}
\put(40,0){\circle*{2}}
\put(40,-6){\scriptsize $x$}
\multiput(0,20)(4,0){9}{\line(1,0){2}}
\put(0,20){\circle*{2}}
\put(-6,20){\scriptsize $y$}
\multiput(40,0)(0,4){5}{\line(0,1){2}}
\put(0,0){\line(2,1){40}}
\put(18,13){\scriptsize $\rho$}
\multiput(20,0)(-1,2){4}{\line(-2,1){1}}
\put(20,3){\scriptsize $\theta$}
\end{picture}
\end{center}

%\vspace*{1mm}

%\pause
{\bf Propri\'et\'e:} Soit $z=x+iy=\rho e^{i\theta}$ un nombre complexe. 

Par définition du sinus et du cosinus, on a
\framebox{\ $x =\rho \cos{\theta}$\ } \quad et \quad 
\framebox{\ $y = \rho \sin{\theta}$\ }

\medskip

{\small On en déduit le résultat suivant.}


{\bf Théorème:} 
$$\boxed{e^{i\theta} = \cos{\theta}+i\sin{\theta}}$$ 
et donc
$$\boxed{\rho e^{i\theta} = \rho\cos{\theta}+i\rho\sin{\theta}}$$
\medskip
{\small {\bf Démonstration:} Pour tout nombre complexe $z=x+iy=\rho e^{i\theta}$, on a vu que $x =\rho \cos{\theta}$ et $y = \rho \sin{\theta}$. Choisissons $\rho=1$ et identifions parties réelles et imaginaires.}
\medskip



{\bf Exemple de conversion en coordonnées cartésiennes:} 
$$2e^{i\pi/4}= 2\cos(\pi/4)+2i\sin(\pi/4) = \sqrt 2+i\sqrt 2$$ 

{\bf Corollaire:} 
$$
\boxed{\cos(\theta)=\frac{e^{i\theta}+e^{-i\theta}}2,\quad \sin(\theta)=\frac{e^{i\theta}-e^{-i\theta}}{2i}}
$$



\end{frame}
%%%%%%%%%%%%%%%%%%%%%%%%%%%%%%%%%%%%%%%%%%%

\begin{frame}%[plain]
%%\only<1>{\addtocounter{framenumber}{-1}}
\frametitle{\bf Des coordonnées cartésiennes aux coordonnées polaires}
\medskip 
\qquad
\begin{center}
\begin{picture}(80,50)(-20,-10)
%\thicklines
\put(-20,0){\vector(1,0){80}}
\put(0,-10){\vector(0,1){50}}
\put(40,20){\circle*{2}}
\put(0,0){\circle*{2}}
\put(-5,-5){\scriptsize $O$}
\put(36,24){\scriptsize $z=x+iy=\rho e^{i\theta}$}
\put(40,0){\circle*{2}}
\put(40,-6){\scriptsize $x$}
\multiput(0,20)(4,0){9}{\line(1,0){2}}
\put(0,20){\circle*{2}}
\put(-6,20){\scriptsize $y$}
\multiput(40,0)(0,4){5}{\line(0,1){2}}
\put(0,0){\line(2,1){40}}
\put(18,13){\scriptsize $\rho$}
\multiput(20,0)(-1,2){4}{\line(-2,1){1}}
\put(20,3){\scriptsize $\theta$}
\end{picture}
\end{center}

%\vspace*{1mm}

%\pause
{\bf Propri\'et\'e:} Soit $z=x+iy=\rho e^{i\theta}$ un nombre complexe non nul. 

Par le théorème de Pythagore, 
$$\boxed{\rho^2=x^2+y^2}$$
{\small\it NB : cette équation donne $\rho$ de manière implicite: il faut encore prendre la racine carrée.}


Une fois connu $\rho$, on détermine un argument $\theta$ de $z$ à l'aide des relations
$$\boxed{\cos{\theta}=\frac x\rho \text{ et } \sin{\theta}=\frac y\rho}$$ 
{\small\it NB : ces équations donnent $\theta$ de manière implicite: il faut encore les résoudre (souvent de tête).}

\medskip

\

{\bf Exemple de conversion en coordonnées polaires:} 
\small
Si $z=\sqrt 2+i\sqrt 2$, alors $\rho^2 = (\sqrt2)^2+(\sqrt2)^2=4$ d'où $\rho=\sqrt 4=2$ et 
$$\cos{\theta}=\frac{\sqrt 2}2 \text{ et } \sin{\theta}=\frac{\sqrt 2}2$$ 
Ainsi, $\theta=\pi/4$ est un argument de $z$ et donc $z=\rho e^{i\theta}=2e^{i\pi/4}$.

\end{frame}
%%%%%%%%%%%%%%%%%%%%%%%%%%%%%%%%%%%%%%%%%%%




\begin{frame}
\frametitle{\bf Exercice: conversion en coordonnées polaires} 
\medskip 

{\bf Exercice:}
{\it Calculer le module et un argument des nombres complexes suivants.}
\vspace*{1mm}

\begin{itemize}
\bitem 
$z=-1-i$
\end{itemize}
\vspace*{2mm}

\pause
{\bf Solution:}
Le module vérifie $\left|z\right|^2= {(-1)^2+(-1)^2}$. Donc $\vert z\vert=\sqrt{2}$.
Un argument est un angle $\theta$ tel que 
$$
\cos{\theta}=\dfrac{-1}{\sqrt{2}} \quad\mbox{et}\quad 
\sin{\theta}=\dfrac{-1}{\sqrt{2}}. 
$$
Ainsi, $\theta = \dfrac{5\pi}{4}$ est un argument de $z$. Et $z=\sqrt 2e^{i\pi/4}$
\vspace*{3mm}

\pause
\begin{itemize}
\bitem 
$z=5-3i$
\end{itemize}
\vspace*{2mm}

\pause
{\bf Solution:}
Le module vérifie\quad 
$\left|z\right|^2= {5^2+3^2}={25+9}={34}$. Donc $\vert z\vert=\sqrt{34}$.
Un argument est un angle $\theta$ tel que 
$$
\cos{\theta}=\dfrac{5}{\sqrt{34}} \quad\mbox{et}\quad 
\sin{\theta}=\dfrac{-3}{\sqrt{34}}. 
$$
De cet angle on peut dire (sans calculette) que  $\tan\theta=-3/5$ et  $z=\sqrt{34}e^{i\theta}$.

\end{frame}

%%%%%%%%%%%%%%%%%%%%%%%%%%%%%%%%%%%%%%%%%%%

\begin{frame}%[plain]
%%\only<1>{\addtocounter{framenumber}{-1}}
\frametitle{\bf Propri\'et\'es du module}
\medskip 

Soit $z=x+iy\in \C$. 
Le {\bf module} de $z$ est donc donné par \framebox{\ $|z|^2= {x^2+y^2}$\ }. 
\vspace*{1mm}

{\bf Propri\'et\'es:} 
\begin{itemize}
\bitem {\ $|z|\geq 0$\ }\quad 
et\quad {\ $|z|=0$ si et seulement si $z=0$\ } 
\vspace*{2mm}

\pause
\bitem si $z=x\in{\mathbb R}$, alors $|z|=\vert x\vert$ est la valeur absolue 
du r\'eel $x$.
\vspace*{2mm}


\pause
\bitem \framebox{\ $z\overline{z} = |z|^2$\ } \quad 
%par cons\'equent\quad  
%\framebox{\ $\dfrac{1}{z}=\dfrac{\overline{z}}{|z|^2}$\ } si $z\neq0$ 
\vspace*{2mm}

\pause
\bitem \framebox{\ $|z_1 z_2|=|z_1||z_2|$\ } \quad 
%et\quad \framebox{\ $\left| \dfrac{1}{z}\right|=\dfrac{1}{|z|}$\ } 
%i $z \neq 0$
\vspace*{2mm}

\pause
\bitem \framebox{\ $|z_1 +z_2|\leq |z_1|+|z_2|$\ } 
\vspace*{1mm}

et $|z_1 +z_2|= |z_1|+|z_2|$ si et seulement si 
$z_1=\lambda\,z_2$ ou $\lambda\in\R_{+}$ 
\vspace*{2mm}

\pause
%\bitem $|z_1\!+\!z_2|\geq ||z_1|\!-\!|z_2||$, \quad i.e.
%$\left\{\begin{array}{l}
%\mbox{$|z_1\!+\!z_2|\geq |z_1|\!-\!|z_2|$ si $|z_1| \geq |z_2|$}, \\ 
%\mbox{$|z_1\!+\!z_2|\geq |z_2|\!-\!|z_1|$ si $|z_1|\leq |z_2|$}. 
%\end{array}\right.$
\end{itemize}

\end{frame}

%%%%%%%%%%%%%%%%%%%%%%%%%%%%%%%%

\begin{frame}
\frametitle{\bf Le cercle unité} 
\medskip 

\begin{center}
\raisebox{-.6\height}{
\begin{picture}(100,100)(-50,-50)
%\thicklines
\put(-50,0){\vector(1,0){100}}
\put(0,-50){\vector(0,1){100}}
\put(0,0){\circle{60}}
\put(30,0){\circle*{2}}
\put(31,-6){\scriptsize $1$}
\put(15,25.98){\circle*{2}}
\put(13,29){$e^{i\theta}$}
\put(15,0){\circle*{2}}
\put(7,-6){\scriptsize $\cos\theta$}
\multiput(0,25.98)(4,0){4}{\line(1,0){2}}
\put(0,25.98){\circle*{2}}
\put(-15,22){\scriptsize $\sin\theta$}
\multiput(15,0)(0,4){7}{\line(0,1){2}}
\end{picture}}
\end{center}
{\bf Propriet\'es:} 
\begin{itemize}

\bitem 
Le {\bf cercle unité} a pour équation polaire $\rho=1$ et pour équation cartésienne $x^2+y^2=1$.
\bitem
\framebox{Le { cercle unité} est composé des nombres $z=e^{i\theta}=\cos\theta+i\sin\theta$}, où $\theta\in\R$ 
\bitem 
\parbox[t]{5.5cm}{
$e^{i\theta}$ est {\bf p\'eriodique} de {\bf p\'eriode $2\pi$}, i.e.}
\quad  
\parbox[t]{3cm}{
\framebox{\ $e^{i(\theta+2k\pi )}=e^{i\theta}$\ } \vspace*{1mm} \\ 
\quad pour tout $k\in\Z$}
\vspace*{2mm}

\pause
\bitem 
$e^{i\theta}$ est un nombre complexe {\bf unitaire}, i.e.\ 
\framebox{\ $|e^{i\theta}|=1$\ }
\vspace*{2mm}
 
\end{itemize}

{\small
{\bf Exemples:}\quad 
$e^{i 0}=e^{i 2\pi}=1$,\quad $e^{i \frac{\pi}{2}}=i$,\quad 
$e^{i \pi}=-1$,\quad $e^{i \frac{3\pi}{2}}=-i$. }
\vspace*{3mm}

{\bf Corollaire:} pour tout angle $\theta\in\R$, $\boxed{\cos^2(\theta) + \sin^2(\theta) = 1.}$ 

\end{frame}
%%%%%%%%%%%%%%%%%%%%%%%%%%%%%%%%%%%%%%%%%%%

\begin{frame}
\begin{center}
3. Polynômes complexes (racines, factorisation). 
\end{center}
\end{frame}
%%%%%%%%%%%%%%%%%%%%%%%%%%%%%%%%%%%%%%%%%%%

\begin{frame}%[plain]
%\only<1>{\addtocounter{framenumber}{-1}}
\frametitle{\bf Puissances de nombres complexes} 
\medskip 

{\bf Rappel:} 
Tout nombre complexe $z$ s'écrit sous
{\bf forme polaire}\ \framebox{\ $z=\rho\ e^{i\theta}$\ }.
\vspace*{2mm}

\quad {\small De plus,\quad 
$z = x+iy = \rho \cos{\theta} + i \rho \sin{\theta} 
=\rho(\cos{\theta} + i \sin{\theta}) = \rho\ e^{i\theta}$.} 
\vspace*{3mm}

\pause
{\bf Calcul des puissances de $z$} 

\begin{itemize}
\bitem 
{puissance enti\`ere positive}: $\boxed{z^n =(\rho e^{i\theta})^n=\rho^n e^{in\theta}},$\ 
o\`u $n\in\N$
\vspace*{1.5mm}

\pause
\bitem 
{ puissance enti\`ere négative}: $\dfrac{1}{z^n}= z^{-n} =  \rho^{-n} e^{-i n\theta}
=\dfrac{1}{\rho^n} e^{-in\theta}$ si $\rho\neq 0$, 
\vspace*{1.5mm}

\pause
\bitem 
\parbox[t]{3cm}{{\bf formule de Moivre}:}
\quad 
\parbox[t]{6.4cm}{
$(e^{i\theta})^n=e^{in\theta}$\quad pour tout $n\in\Z$, donc 
\vspace*{2mm} \\ 
$\boxed{(\cos \theta +i \sin \theta)^n =\cos (n\theta) +i \sin (n\theta)}$} 
\end{itemize}

\end{frame}

%%%%%%%%%%%%%%%%%%%%%%%%%%%%%%%%%%%%%%%%%%%%

\begin{frame}
\frametitle{\bf Racines deuxièmes et troisièmes d'un nombre complexe} 
\medskip 

{\bf Définition:} On appelle {\bf racine deuxième} du nombre complexe $z$ tout nombre $w$ tel que
$$
w^2=z
$$
{\bf Exemple:} Soit $z=-1$. 

Comme $-1=e^{i\pi}$, 
$$w_1=e^{i\pi/2}=i \quad \text{est racine deuxième de $z=-1$.}$$ 
Comme $-1=e^{i(\pi+2\pi)}=e^{i3\pi}$, 
$$w_2=e^{i3\pi/2}=-i\quad \text{est aussi racine deuxième de $z=-1$.}$$

{\bf Exercice:} Dans le cas particulier où $z=x\in\R_+$, les racines deuxièmes de $z$ sont les racines carrées de $x$: $w_1=\sqrt x$ et $w_2=-\sqrt x$. Que sont les racines deuxièmes de $z=x\in\R_{-}$ ?

\

{\bf Définition:} On appelle {\bf racine troisième} du nombre complexe $z$ tout nombre $w$ tel que
$$
w^3=z
$$
{\bf Exercice:} Vérifier que $w_1=e^{i\pi/3}$, $w_2=e^{i\pi}=-1$ et $w_3=e^{i5\pi/3}$ sont racines troisièmes de $z=-1$.
\end{frame}

%%%%%%%%%%%%%%%%%%%%%%%%%%%%%%%%%%%%%%%%%%%


\begin{frame}
\frametitle{\bf Racines deuxièmes et troisièmes d'un nombre complexe} 
\medskip 

{\bf Théorème:} 
\begin{itemize}
\bitem
Un nombre complexe $z=\rho e^{i\theta}$ non nul a toujours \underline{deux} {racines deuxièmes} 
$w_1$ et $w_2$.
Comme $\rho e^{i\theta}=\rho e^{i(\theta+2\pi)}$, les  racines deuxièmes de $z$ sont
$$\boxed{w_1={\rho^{\frac12}}\ e^{i\frac{\theta}{2}},\quad\text{et}\quad w_2=\rho^{\frac12}\ e^{i\left(\frac{\theta}{2}+\pi\right)} }.$$
et on a $\boxed{w_2=-w_1}$.

\hfill
\raisebox{-.6\height}{
\begin{picture}(75,50)(-45,-25)
\setlength{\unitlength}{.6pt}
\put(-50,0){\vector(1,0){100}}
\put(0,-50){\vector(0,1){100}}
\put(0,0){\circle{60}}
\put(30,0){\circle*{4}}
\put(31,-13){\scriptsize $\sqrt{\rho}$}
\put(15,25.98){\circle*{4}}
\put(13,29){\scriptsize $w_1$}
\put(-15,-25.98){\circle*{4}}
\put(-25,-38){\scriptsize $w_2$}
\put(-15,-25.98){\line(1,1.73){30}}
\put(12,6){\scriptsize $\frac{\theta}{2}$}
\end{picture}}
\vspace*{3mm}

\pause
\bitem
Un nombre complexe $z=\rho e^{i\theta}$ non nul a toujours \underline{trois} { racines troisièmes} 
$w_1$, $w_2$ et $w_3$.
$$\boxed{w_1=\rho^{1/3} e^{i\frac{\theta}{3}}, \quad w_2=\rho^{1/3} e^{i(\frac{\theta}{3}+\frac{2\pi}{3})}\quad\text{et}\quad w_3=\rho^{1/3} e^{i(\frac{\theta}{3}+\frac{4\pi}{3})} }.$$
Si une racine troisième est r\'eelle, les deux autres sont conjugu\'ees 
complexes.  
\hfill
\parbox[t]{5.7cm}{
{\small 
{\bf Exemple:} Si $\theta=0$. Alors\ $w_1=\sqrt[3]{\rho}$ et
$$ 
w_2=\sqrt[3]{\rho}\ e^{i\frac{2\pi}{3}}\quad \mbox{et}\quad  
w_3=\sqrt[3]{\rho}\ e^{i\frac{4\pi}{3}}=\overline{w_2}
$$}}
\quad
\raisebox{-.6\height}{
\begin{picture}(100,50)(-50,-25)
\setlength{\unitlength}{.6pt}
\put(-50,0){\vector(1,0){100}}
\put(0,-50){\vector(0,1){100}}
\put(0,0){\circle{60}}
\put(30,0){\circle*{4}}
\put(31,-13){\scriptsize $w_1$}
\put(-15,25.98){\circle*{4}}
\put(-25,32){\scriptsize $w_2$}
\put(-15,-25.98){\circle*{4}}
\put(-25,-38){\scriptsize $w_3$}
\put(-15,-25.98){\line(0,1){50}}
\put(-15,-25.98){\line(1.73,1){47}}
\put(-15,25.98){\line(1.73,-1){47}}
\end{picture}}

\end{itemize}

\end{frame}

%%%%%%%%%%%%%%%%%%%%%%%%%%%%%%%%%%%%%%%%%%%


\begin{frame}%[plain]
%\only<1>{\addtocounter{framenumber}{-1}}
\frametitle{\bf Exercice: racines deuxièmes de nombres complexes} 
\medskip 

{\bf Exercice:}
{\it Calculer les racines deuxièmes des complexes suivants:} 
\vspace*{2mm}

\begin{itemize}
\bitem 
$z=1+i\sqrt{3}$ 
\end{itemize}

\pause
{\bf Solution:} 
On \'ecrit $z$ en coordonnées polaires: $z=1+i\sqrt{3}=2 e^{i\frac{\pi}{3}}$. 
Alors,
\begin{align*}
w_1 & =\sqrt{2}\ e^{i\frac{\pi}{6}}=\frac{\sqrt{2}}{2}(\sqrt{3}+i) \\
w_2 & =\sqrt{2}\ e^{i\left(\frac{\pi}{6}+\pi\right)}
=-\frac{\sqrt{2}}{2}(\sqrt{3}+i)
\end{align*} 

\pause
\begin{itemize}
\bitem 
$z=15-8i$
\end{itemize}

\pause
{\bf Solution:} 
Comme on ne connait pas la forme polaire de $z$, on revient à la définition d'une racine deuxième en cherchant 
$w=x+iy$ tel que $w^2=z$. Soit
$$(x+iy)^2=(x^2-y^2)+2xyi =15-8i$$ 

 Astuce: pour résoudre on commence par calculer $\vert w\vert^2=\vert z\vert$.
$$\left|w^2\right|=x^2+y^2=\left|z\right|=\sqrt{225+64}=17.$$
De sorte que
\begin{align*}
& \left\{\begin{array}{l} 
x^2-y^2 = 15 \\ 2xy = -8 \\ x^2+y^2 = 17 
\end{array}\right. 
\ \Leftrightarrow\
\left\{\begin{array}{l} 
y^2 = 1 \\ x = -\frac{4}{y} 
\end{array}\right. 
\ \Leftrightarrow\ 
\left\{\begin{array}{l} 
y = 1 \\ x = -4
\end{array}\right. 
\mbox{ou}
\left\{\begin{array}{l} 
y = -1 \\ x = 4
\end{array}\right. 
\end{align*}
On a donc\quad $w_1=-4+i$\quad et\quad $w_2=4-i$. 

\end{frame}


%%%%%%%%%%%%%%%%%%%%%%%%%%%%%%%%%%%%%%%%%%%%%%%%%%
%%%%%%%%%%%%%%%%%%%%%%%%%%%%%%%%%%%%%%%%%%%%%%%%%%

\subsection{Polyn\^omes} 

\begin{frame}
\frametitle{\bf Polyn\^omes complexes et racines}
\medskip 

{\bf D\'efinition:} 
Un {\bf polyn\^ome complexe} est une expression de la forme 
$$P(z)=a_n z^n + a_{n-1}z^{n-1} +\cdots+ a_1 z + a_0,$$ 
où les nombres $a_0, a_1,...,a_n$ sont des paramètres complexes et $z$ est une variable complexe. 
\vspace*{1mm} 

Le {\bf degr\'e} de $P(z)$ est le plus grand entier $k$ tel que 
$a_k\neq 0$. 
\vspace*{2mm} 

{\small
{\bf Exemple:}
\quad
\parbox[t]{7cm}{
$z^3+(5-i)z$\quad est un polyn\^ome de degr\'e $3$ 
%$i\sin(X^3)+5X$\quad n'est pas un polyn\^ome }
}}

\

{\bf D\'efinition:} 
Une {\bf racine} d'un polyn\^ome complexe $P(z)$ est un nombre 
complexe $z$ tel que $$P(z)=0$$ 
\vspace*{1mm} 

{\small
{\bf Exemple:} 
\quad $z=i$\ et\ $z=-i$\ sont racines de $P(z)=z^2\!+\!1$.
}

\end{frame}

%%%%%%%%%%%%%%%%%%%%%%%%%%%%%%%%

\begin{frame}%[plain]
%\only<1>{\addtocounter{framenumber}{-1}}
\frametitle{\bf Racines des polyn\^omes} 
\medskip 

\vspace*{1mm} 

\pause
{\bf Lemme:} 
Si $z_1$ est une racine de $P(z)$, alors il existe un entier $m_1\geq 1$ 
et un polyn\^ome $Q(z)$ tels que 
$$
P(z)=(z-z_1)^{m_1} Q(z)\quad\mbox{et}\quad Q(z_1)\neq 0.
$$ 
On appelle $m_1$ la {\bf multiplicit\'e} de la racine $z_1$. 
\vspace*{1mm} 

\

\pause
{\bf Th\'eor\`eme de d'Alembert-Gauss:} 
Tout polyn\^ome complexe $P(z)$ de degr\'e $n$ peut s'écrire sous la forme 
$$ 
P(z) = a_n (z-z_1)^{m_1} \cdots (z-z_k)^{m_k} 
$$
o\`u $a_n\in\C$, $z_1,...,z_k$ sont les racines distinctes de $P(z)$ 
et\ où $m_1+\cdots+m_k=n$. 
\vspace*{1mm} 

\

\pause
{Par cons\'equent, {\bf tout polyn\^ome complexe de degr\'e $n$ admet $n$ racines} (qui ne sont pas forcément distinctes).}  

\

{\bf Exemple:} Le polynôme $P(z)=z^2+1$ a pour racines $z_1=i$ et $z_2=-i$. $P$ est de degré $2$, ce qui force $m_1=m_2=1$. Enfin, par définition de $P(z)$, $a_2=1$. Par le théorème,  on conclut que $$z^2+1= (z-i)(z+i)$$

\end{frame}

%%%%%%%%%%%%%%%%%%%%%%%%%%%%%%%%

\begin{frame}
\frametitle{\bf Racines d'un polyn\^ome complexe de degr\'e $2$} 
\medskip 

{\bf Proposition:} 
Les solutions de l'\'equation\ \framebox{\ $az^2+bz+c=0$\ } 
\`a coefficients complexes sont 
$$
\mbox{\framebox{\ $z = \dfrac{-b \pm \delta}{2a} \in \C$\ }}
$$ 
o\`u $\delta\in \C$ est une racine deuxième du discriminant 
$\Delta = b^2 - 4ac \in \C$, c'est-\`a-dire un nombre complexe tel que 
$\delta^2=\Delta$. 
\vspace*{4mm}

\pause
{Par cons\'equent, } 
le polyn\^ome $P(z)=az^2+bz+c$ possède 
\begin{itemize}
\bitem 
\underline{une racine double}\quad $z=-b/2a$\ \underline{si $\Delta=0$}, 

\bitem 
\underline{deux racines distinctes}\quad 
$z_1 = \dfrac{-b + \delta}{2a}$\ et\ $z_2 = \dfrac{-b - \delta}{2a}$\
\underline{si $\Delta\neq 0$.} 
\end{itemize}


\end{frame}

%%%%%%%%%%%%%%%%%%%%%%%%%%%%%%%%

\begin{frame}%[plain]
%\only<1>{\addtocounter{framenumber}{-1}}
\frametitle{\bf Exercice: \'equation complexe de degr\'e $2$} 
\medskip 

{\bf Exercice:} 
{\it R\'esoudre l'\'equation\quad $z^2-(1+i)z+6-2i=0$.}
\vspace*{1.5mm}

\pause
{\bf R\'eponse:} 
Les solutions de cette \'equation sont\quad 
$z=\dfrac{1+i\pm\delta}{2}$, o\`u il faut trouver $\delta=x+iy$ 
tel que $\delta^2=\Delta$ et  
$\left|\delta^2\right|=\left|\Delta \right|$, 
\pause 
c'est-\`a-dire:
\begin{align*}
(x^2-y^2)+i\,2xy & = (1+i)^2-4(6-2i) \\ 
& = 1+2i-1-24+8i = -24+10i \\ 
x^2+y^2 & = 2 \left|-12+5i \right| = 2\sqrt{144+25} = 2\times 13 =26
\end{align*}
\pause 
On a alors
$$
\left\{\begin{array}{l} 
x^2-y^2 = -24 \\ 2xy=10 \\ x^2+y^2=26 
\end{array}\right. 
\quad \Leftrightarrow\quad 
\left\{\begin{array}{l} 
2x^2 = 2 \\ xy=5 
\end{array}\right. 
\quad \Leftrightarrow\quad 
\left\{\begin{array}{l} 
x = \pm 1 \\ y=5/x=\pm 5
\end{array}\right. 
$$
et par cons\'equent\quad $\delta = \pm (1+5i)$. 
\pause
On a donc 
\begin{align*}
z_1 & =\dfrac{1+i+(1+5i)}{2} = \dfrac{2+6i}{2}=1+3i \\ 
z_2 & =\dfrac{1+i-(1+5i)}{2} = \dfrac{0-4i}{2}=-2i
\end{align*}

\end{frame}

%%%%%%%%%%%%%%%%%%%%%%%%%%%%%%%%

\begin{frame}
\frametitle{\bf Racines complexes d'un polyn\^ome r\'eel} 
\medskip 

{\bf Proposition:}
Si $P(z)$ est un polyn\^ome \`a coefficients \underline{r\'eels}, et $z_1\in\C$ 
est une racine complexe de $P(z)$ de multiplicit\'e $m_1$, 
alors sa conjugu\'ee $z_2=\overline{z_1}$ est aussi une racine de $P(z)$, 
de m\^eme multiplicit\'e $m_2=m_1$. 
\vspace*{4mm}

{\small
{\bf Exemple:} Le polyn\^ome à coefficients r\'eels $P(z)=z^2+1$ a pour racines $z_1=i$ et $z_2=-i$. 
}
\vspace*{4mm}

\pause
{Puisque $z=\overline{z}$ si seulement si $z\in\R$,}  
\begin{itemize}
\bitem 
Tout polyn\^ome r\'eel de degré 2 tel que  $\Delta<0$ admet {deux racines complexes conjugu\'ees} $z_1$ et $z_2=\overline{z_1}$. 

\bitem  
Tout polyn\^ome r\'eel {de degr\'e $3$} 
admet {au moins une racine r\'eelle}.
\end{itemize}
\vspace*{4mm}

{\small
{\bf Exemple:} Le polyn\^ome $P(z)=z^2-4z+5$ est tel que $\Delta=16-20<0$. Il a deux racines complexes 
$z_1=2\!+\!i$ et $z_2=2\!-\!i=\overline{2\!+\!i}$. 
}
\end{frame}

%%%%%%%%%%%%%%%%%%%%%%%%%%%%%%%%%%%%%%%%%%%%%%%%%%%%%%%%%%%%%%%%%%%%%
%%%%%%%%%%%%%%%%%%%%%%%%%%%%%%%%%%%%%%%%%%%%%%%%%%%%%%%%%%%%%%%%%%%%%

