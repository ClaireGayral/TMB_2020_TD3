%
%   TMB-diapo2.tex  2019-2020  A. Mikelic & L. M. Tine  \` \^ 
%
%%%%%%%%%%%%%%%%%%%%%%%%%%%%%%%%%%%%%%%%%%%%%%%%%%%%%%%%%%%%%%%%%%%%%%

%%%%%%%%%%%%%%%%%%%%%%%%%%%%%%%%%%%%%%%%%%%%%%%%%%%%%%%%%%%%%%%%%%%%

\documentclass[10pt]{beamer}
% \documentclass[8pt,handout]{beamer} % to get handout of the presentation

%\usetheme[height=1cm,width=1.7cm]{}%{sidebar}	

%	
% packages 
%
\usepackage{amsfonts,amsmath,amssymb,amscd,amstext,mathabx,bbm}
\usepackage{textcomp,multicol,multirow,enumitem}
\usepackage{pict2e,graphicx}
\usepackage{mathrsfs}


\usepackage[utf8]{inputenc}
\usepackage[francais]{babel}  % bof 




%%%%%%%%%%############%%%%%%%%%%%%%%%##############%%%%%%%%%%%%%%%##########
\renewcommand{\contentsname}{Programme du cours}

\newcommand{\bitem}{\item[$\bullet$]}
\newcommand{\litem}{\hspace{-.5cm}\item[$\bullet$]}
\setlist[itemize]{leftmargin=*}

\newcommand{\ds}{\displaystyle}

\newcommand{\N}{\mathbb N}
\newcommand{\Z}{\mathbb Z}
\newcommand{\Q}{\mathbb Q}
\newcommand{\R}{\mathbb R}
\newcommand{\C}{\mathbb C}

\newcommand{\lra}{\longrightarrow}

\newcommand{\Arg}{\mathrm{Arg}}
\renewcommand{\Re}{\mathrm{Re}}
\renewcommand{\Im}{\mathrm{Im}}

\newcommand{\dist}{\mathrm{dist}}
\newcommand{\Vect}{\mathrm{Vect}} 
\newcommand{\Lin}{\mathrm{Lin}}  
\newcommand{\Rot}{\mathrm{Rot}} 
\newcommand{\Ref}{\mathrm{Ref}} 
\newcommand{\calF}{{\cal F}}
\newcommand{\calL}{{\cal L}}
\newcommand{\calM}{{\cal M}}
\newcommand{\calS}{{\cal S}}
\newcommand{\matun}{\mathbbm{1}}
\newcommand{\Aire}{\mathrm{Aire}}
\newcommand{\Vol}{\mathrm{Vol}} 
\newcommand{\id}{\mathrm{id}}

\newcommand{\vi}{\vec{\mbox{\it \i}}\,}
\newcommand{\vj}{\vec{\mbox{\it \j}}\,} 
\newcommand{\vk}{\vec{\mbox{\it k}}\,} 
\newcommand{\ve}{\vec{e}}
\newcommand{\vv}{\vec{v}}
\newcommand{\vu}{\vec{u}}
\newcommand{\vw}{\vec{w}}
\newcommand{\vx}{\vec{x}}
\newcommand{\vy}{\vec{y}}
\newcommand{\vXY}[2]{\left(\!\!\begin{array}{c} #1 \\ #2 \end{array}\!\!\right)}
\newcommand{\vXYZ}[3]{\left(\!\!\begin{array}{c} #1 \\ #2 \\ #3 \end{array}\!\!\right)}

\newcommand{\vA}{\overrightarrow{A}}
\newcommand{\vB}{\overrightarrow{B}}
\newcommand{\vE}{\overrightarrow{E}}
\newcommand{\vF}{\overrightarrow{F}}
\newcommand{\vG}{\overrightarrow{{\cal G}}}
\newcommand{\vV}{\overrightarrow{V}}
\newcommand{\vU}{\overrightarrow{U}}
\newcommand{\vdl}{\overrightarrow{d\ell}}
\newcommand{\vdS}{\overrightarrow{dS}}

\newcommand{\ch}{\mathrm{ch}\,} 
\newcommand{\sh}{\mathrm{sh}\,} 
\renewcommand{\th}{\mathrm{th}\,} 

\newcommand{\ddx}{\frac{d}{dx}}
\newcommand{\dd}[1]{\frac{d #1}{dx}}

%%%%%%%%%%%%%%%%%%%%%%%%%%%%%%%%%%%%%%%%%%%%%%%%%%%%%%%%%%%%%%%%%%%%%%%%

%\newenvironment{definition}{\medskip \noindent{\bf D\'efinition.}\ }
%{\medskip}
\newenvironment{theoreme}{\medskip \noindent{\bf Th\'eor\`eme.}\ }
{\medskip}
\newenvironment{theoreme-nom}[1]{\medskip \noindent{\bf Th\'eor\`eme #1.}\ }
{\medskip}
\newenvironment{proposition}{\medskip \noindent{\bf Proposition.}\ }
{\medskip}
\newenvironment{corollaire}{\medskip \noindent{\bf Corollaire.}\ }
{\medskip}
\newenvironment{lemme}{\medskip \noindent{\bf Lemme.}\ }
{\medskip}
\newenvironment{remarque}{\medskip \noindent{\bf Remarque.}\ }{\medskip}
\newenvironment{rappel}{\medskip \noindent{\bf Rappel.}\ }{\medskip}
\newenvironment{exercice}{\medskip \noindent{\bf Exercice.}\ }{\medskip}
\newenvironment{exercices}{\medskip \noindent{\bf Exercices.}\ }{\medskip}
%\newenvironment{exemple}{\medskip \begin{quote}\noindent{\bf Exemple.}\quad}%
%     {\end{quote}\medskip}
%\newenvironment{exemples}{\medskip \begin{quote}\noindent{\bf Exemples.}\quad}%
%     {\end{quote}\medskip}
\newenvironment{exemple}{\medskip \noindent{\bf Exemple.}\ }{\medskip}
\newenvironment{exemples}{\medskip \noindent{\bf Exemples.}\ }{\medskip}
\newenvironment{preuve}{\medskip \noindent{\bf Preuve.}\ }%
{\medskip}
\newenvironment{application}{\medskip \noindent{\bf Application.}\ }%
{\medskip}
\newenvironment{question}{\medskip \noindent{\bf Question.}\ }{\medskip}
\newenvironment{conclusion}{\medskip \noindent{\bf Conclusion.}\ }{\medskip}
%%%%%%%%%%%############%%%%%%%%%%%%############ù%%%%%%%%%%ù
















%\setlength{\textwidth}{17cm}
%\setlength{\oddsidemargin}{-1.05cm}
%\setlength{\evensidemargin}{-1.05cm}
%
% packages pour dessins avec tikzpicture
%
%\usepackage{pgf,tikz,pgfplots}
%\pgfplotsset{compat=1.15}
%\usepackage{mathrsfs}
%\usetikzlibrary{arrows}
\usepackage{tikz}
\usepackage{tikz-3dplot}

% \usepackage{cours/TMB-dessins}

%
% macros
%
% \usepackage{cours/TMB-macros}
%\usepackage[paperheight=11cm, paperwidth=8.5cm]{geometry}
%
% macros pour beamer 
%
\newcommand{\uv}{\mbox{\small $u$,$v$}}
\newcommand{\xy}{\mbox{\small $x$,$y$}}
\newcommand{\xxyy}{\mbox{\footnotesize $\sqrt{x^2\!+\!y^2}$}}

%\AtBeginSubsection[]
%{
%  \begin{frame}<beamer>
%    \frametitle{Plan}
%    \tableofcontents[currentsection,currentsubsection]
%  \end{frame}
%}

\setbeamertemplate{navigation symbols}{} 

\usepackage[footheight=.5em]{beamerthemeboxes}
\addfootboxtemplate{\color{white}}{\color{black}\hfill\large{\insertframenumber}\hspace{1em}\null} %/\inserttotalframenumber}\hspace{1em}\null}

%%%%%%%%%%%%%%%%%%%%%%%%%%%%%%%%%%%%%%%%%%%%%%%%%%%%%%%%%%%%%%%%%%%%%
%%%%%%%%%%%%%%%%%%%%%%%%%%%%%%%%%%%%%%%%%%%%%%%%%%%%%%%%%%%%%%%%%%%%%

\begin{document}
%%%%%%%%%%%%%%%%%%%%%%%%%%%%%%%%%%%%%%%%%%%%%%%%%%%%%%%%%%%%%%%%%%%%%
%%%%%%%%%%%%%%%%%%%%%%%%%%%%%%%%%%%%%%%%%%%%%%%%%%%%%%%%%%%%%%%%%%%%%
\section{2 Fonctions}

\begin{frame}[plain]
\only<1>{\addtocounter{framenumber}{-1}}
\frametitle{\bf TMB -- Chapitre 2 \\ 
Fonctions d'une variable r\'eelle}

Dans ce chapitre: 
\begin{itemize}
\item[1.] Fonctions usuelles. 

\item[2.] Graphes des fonctions. 

\item[3.] Fonctions croissantes, d\'ecroissantes, monotones. 
Fonctions paires et impaires. 

\item[4.] Op\'erations entre fonctions: addition, multiplication, 
composition. 

\item[5.] Fonctions r\'eciproques. 
\end{itemize}

\end{frame}

%%%%%%%%%%%%%%%%%%%%%%%%%%%%%%%%%%%%%%%%%%%%%%%%%%
%%%%%%%%%%%%%%%%%%%%%%%%%%%%%%%%%%%%%%%%%%%%%%%%%%

\subsection{Fonctions usuelles} 

\begin{frame}
\frametitle{\bf 1. Fonctions r\'eelles}
\medskip 

{\bf D\'efinition:} 
Une {\bf fonction (r\'eelle)} est une ``loi'' qui associe 
\`a tout $x\in\R$ \underline{au maximum une valeur} $y\in\R$, 
qu'on note $y(x)$ car elle d\'epend de $x$. Une fonction est aussi not\'ee 
$$\mbox{\framebox{\ 
$f:\R\longrightarrow \R,\ x\mapsto y=f(x)$
\ }} $$
\vspace*{-5mm}

\pause
\begin{itemize}
\bitem
Le {\bf domaine (de d\'efinition)} d'une fonction $f$ est l'ensemble 
\vspace*{-1mm}
$$\mbox{\framebox{\ 
$D_f=\big\{ x\in\R\ |\ f(x)\in\R,\ 
\mbox{i.e. la valeur $f(x)$ est bien d\'efinie}\Big\}$
\ }} $$
\vspace*{-3mm}

\bitem
L'{\bf image} d'une fonction $f$ est l'ensemble 
\vspace*{-1mm}
$$\mbox{\framebox{\ 
$I_f=\big\{ y\in\R\ |\ y=f(x)\ \mbox{pour un $x\in D_f$} \Big\}$
\ }} $$
\end{itemize}
\vspace*{1mm}

\pause
\underline{Attention}: une loi qui associe \`a $x\in\R$ 
\underline{deux} valeurs distinctes $y_1,y_2\in\R$ (ou plus) 
n'est pas une fonction.  
\vspace*{2mm}

\pause
{\small 
{\bf Exemples:}
\begin{itemize}
\bitem
La loi $f(x)=x^2$ est une fonction de domaine $D_f\!=\!\R$ et image $I_f\!=\!\R^+$.
\bitem
La loi $f(x)=\sqrt{x}$ est une fonction de domaine $D_f\!=\!\R^+$ et image $I_f\!=\!\!\R^+$.
\bitem
La loi $f(x)=\pm\sqrt{x}$ n'est pas une fonction, ex. $f(4)=+2$ ou bien $-2$. 
\end{itemize}
}

\end{frame}

%%%%%%%%%%%%%%%%%%%%%%%%%%%%%%%%%%%%%%%%%%%%%%%%%%

\begin{frame}[plain]
\only<1>{\addtocounter{framenumber}{-1}}
\frametitle{\bf Polyn\^omes, fractions, racines}
\medskip 

{\bf D\'efinition:} 
On appelle ``usuelles'' les fonctions $f:\R\rightarrow\R$ suivantes:
\begin{itemize}
\bitem
{\bf fonctions polynomiales}, abr\'eg\'e en ``polyn\^omes'': 
$$
\mbox{\framebox{\ $f(x)=a_o+a_1\,x+a_2\,x^2+\cdots +a_n\,x^n$\ }}
\quad a_i\in\R,
\ \mbox{avec\ \framebox{\ $D_f=\R$\ }.} 
$$

\pause
\bitem
{\bf fractions rationnelles}, abr\'eg\'e en ``fractions'': ce sont les 
quotients de polyn\^omes $a(x)$ et $b(x)$
$$
\mbox{\framebox{\ $f(x)=\frac{a(x)}{b(x)}$\ }\quad 
avec\ \framebox{\ $D_f=\Big\{ x\in\R\ |\ b(x)\neq 0 \Big\}$\ }.}  
$$

\pause
\bitem
{\bf fonctions radicales}, abr\'eg\'e en ``racines'': ce sont les 
racines $k$-i\`emes de polyn\^omes $a(x)$, pour $k\in\N$
$$
\mbox{\framebox{\ $f(x)=\sqrt[k]{a(x)}$\ }\quad 
d\'efini par\quad $f(x)^k=a(x)$}
$$
avec domaine\ \framebox{\  
$D_f=\left\{ x\in\R\ |\ \Big\langle\begin{array}{ll} 
a(x)\in\R &\mbox{si $k$ est impair} \\ a(x)\in \R^+ &\mbox{si $k$ est pair} 
\end{array}\right\}$\ }. 
\end{itemize}

\end{frame}

%%%%%%%%%%%%%%%%%%%%%%%%%%%%%%%%%%%%%%%%%%%%%%%%%%

\begin{frame}
\frametitle{\bf Fonctions circulaires}
\medskip 

\begin{itemize}
\parbox{10cm}{
\bitem
{\bf cosinus}\ \framebox{\ $f(x)\!=\!\cos x$\ }\quad avec\quad  
\framebox{\ $D_{\cos}\!=\!\R$\ }\quad et\quad \framebox{\ $I_{\cos}\!=\![-1,1]$\ }
}
\vspace*{3mm}

\bitem
{\bf sinus}\ \framebox{\ $f(x)\!=\!\sin x$\ }\quad avec\quad 
\framebox{\ $D_{\sin}\!=\!\R$\ }\quad et\quad \framebox{\ $I_{\sin}\!=\![-1,1]$\ }
\vspace*{3mm}

\parbox[t]{6cm}{
o\`u $(\cos x,\sin x)$ sont les coordonn\'ees du point $P$ 
qui se trouve sur le cercle unitaire \`a angle $x$ 
mesur\'e dans le sens antihoraire depuis l'axe de direction $\vi$. 
\vspace*{1mm}

Puisque le cercle a \'equation $X^2+Y^2=1$, si on pose $X=\cos x$ et 
$Y=\sin x$ on a
$$
\mbox{\framebox{\ 
$\cos^2 x + \sin^2 x = 1$\ } \quad pour tout $x\in\R$.}
$$
}
\quad\raisebox{-\height}{
\begin{picture}(90,90)(-45,-45)
\setlength{\unitlength}{1.5pt}
\put(-30,0){\vector(1,0){60}}
\put(28,-5){\scriptsize{$\vi$}}
\put(0,-30){\vector(0,1){60}}
\put(-5,28){\scriptsize{$\vj$}}
\put(0,0){\circle{40}}
\put(20,-20){\vector(0,1){50}}
%
\put(14,14){\circle*{2}}
\put(12,17){\scriptsize{$P$}}
\put(0,0){\line(1,1){20}}
\put(8,0){\vector(-2,3){3}}
\put(8,3){\scriptsize{$x$}}
\multiput(14,0)(0,4){4}{\line(0,1){2}}
\multiput(0,14)(4,0){4}{\line(1,0){2}}
%
\linethickness{0.5mm}
\put(0,0){\line(1,0){14}}
\put(2,-5){\scriptsize{$\cos x$}}
\put(0,0){\line(0,1){14}}
\put(-12,8){\scriptsize{$\sin x$}}
\put(20,0){\line(0,1){20}}
\put(22,10){\scriptsize{$\tan x$}}
\end{picture}
}

\pause
\bitem
{\bf tangente}\ \framebox{\ $f(x)=\tan x=\dfrac{\sin x}{\cos x}$\ }\quad 
avec\quad 
$$
\mbox{
\framebox{\ $D_{\tan}=\{ x\in\R\ |\ x\neq \frac{\pi}{2}+k\pi,\ k\in\Z\}$}\quad 
et\quad \framebox{\ $I_{\tan}=\R$\ }
}
$$
\end{itemize}

\end{frame}

%\begin{frame}
%\frametitle{\bf Formulaire sur les fonctions circulaires}

%\centering{Voir livret de cours}

%\end{frame}


%%%%%%%%%%%%%%%%%%%%%%%%%%%%%%%%%%%%%%%%%%%%%%%%%%

\begin{frame}[plain]
\only<1>{\addtocounter{framenumber}{-1}}
\frametitle{\bf Fonctions arc}
\medskip 

\begin{itemize}
\bitem
{\bf arccosinus}\ \framebox{\ $f(x)\!=\!\arccos x$\ }\quad 
\vspace*{1mm}

$\arccos x$ est l'angle compris entre $0$ et $\pi$ qui a $x$ comme cosinus, 
c.-\`a-d.\quad $\arccos x = \theta \ \Leftrightarrow\ x=\cos \theta$\quad 
et\quad $\theta\in [0,\pi]$, alors
$$
\mbox{
\framebox{\ $D_{\arccos}\!=\![-1,1]$\ }\quad et\quad 
\framebox{\ $I_{\arccos}\!=\![0,\pi]$\ }
}
$$
\vspace*{1mm}

\pause
\bitem
{\bf arcsinus}\ \framebox{\ $f(x)\!=\!\arcsin x$\ }\quad 
\vspace*{1mm}

$\arcsin x$ est l'angle compris entre $-\frac{\pi}{2}$ et $\frac{\pi}{2}$ 
qui a $x$ comme sinus, c.-\`a-d.\quad 
$\arcsin x = \theta \ \Leftrightarrow\ x=\sin \theta$\quad 
et\quad $\theta\in [-\frac{\pi}{2},\frac{\pi}{2}]$, alors\quad 
$$
\mbox{
\framebox{\ $D_{\arcsin}\!=\![-1,1]$\ }\quad et\quad 
\framebox{\ $I_{\arcsin}\!=\![-\frac{\pi}{2},\frac{\pi}{2}]$\ }
}
$$
\vspace*{1mm}

\pause
\bitem
{\bf arctangente}\ \framebox{\ $f(x)=\arctan x$\ }\quad 
\vspace*{1mm}

$\arctan x$ est l'angle compri entre $-\frac{\pi}{2}$ et $\frac{\pi}{2}$ 
qui a $x$ comme tangente, c.-\`a-d.\quad 
$\arctan x = \theta \ \Leftrightarrow\ x=\tan \theta$\quad 
et\quad $\theta\in ]-\frac{\pi}{2},\frac{\pi}{2}[$, alors\quad 
$$
\mbox{
\framebox{\ $D_{\arctan}=\R$}\quad et\quad 
\framebox{\ $I_{\arctan}=]-\frac{\pi}{2},\frac{\pi}{2}[$\ }
}
$$
\end{itemize}

\end{frame}

%%%%%%%%%%%%%%%%%%%%%%%%%%%%%%%%%%%%%%%%%%%%%%%%%%

\begin{frame}
\frametitle{\bf Exponentiel}
\medskip 

\begin{itemize}
\bitem
La {\bf fonction exponentielle}, abr\'eg\'e en ``exponentiel'', \\ 
de base le {\bf nombre de N\'eper} 
{\small (de Euler et Napier, XVII s.)}
\vspace*{-2mm}
$$
e=\lim_{n\rightarrow\infty} \left(1+\frac{1}{n}\right)^n 
= \sum_{n=0}^\infty \frac{1}{n!} \simeq 2,7182
\vspace*{-2mm}
$$
est la fonction\ \framebox{\ $f(x)=e^x = \exp(x)$\ }\ qui peut \^etre d\'efinie 
de plusieurs fa\c{c}ons (voir les prochains chapitres pour comprendre): 
\vspace*{2mm}

\begin{itemize}
\item[i)]
c'est la seule fonction continue qui \underline{transforme une somme en} 
\underline{produit}, $\exp(x+y)=\exp(x)\, \exp(y)$, et qui vaut $e$ en $x=1$;
\vspace*{3mm}

\item[ii)]
c'est la seule solution de l'\underline{\'equation diff\'erentielle}\quad 
$f'(x)=f(x)$ \\ qui vaut $1$ en $x=0$;
\vspace*{3mm}

\item[iii)]
comme \underline{somme de s\'erie}\quad 
$\ds e^x=\sum_{n=0}^\infty \frac{1}{n!}\,x^n$.
\end{itemize}
\vspace*{3mm}

On a\quad \framebox{\ $D_{\exp}=\R$\ }\quad et\quad 
\framebox{\ $I_{\exp}= ]0,\infty[$\ }. 
\end{itemize}

\end{frame}

%%%%%%%%%%%%%%%%%%%%%%%%%%%%%%%%%%%%%%%%%%%%%%%%%%

\begin{frame}[plain]
\only<1>{\addtocounter{framenumber}{-1}}
\frametitle{\bf Logarithme}
\medskip 

\begin{itemize}
\bitem
La {\bf fonction logarithme naturel}, abr\'eg\'e en ``logarithme'', 
est la fonction\ \framebox{\ $f(x)=\ln x$\ }\ qui donne l'exposant \`a 
l'exponentiel pour obtenir $x$, c'est-\`a-dire telle que\quad 
$\ln x = y \quad\Leftrightarrow\quad e^y=x$. 
\vspace*{3mm}

\pause
Elle peut \'egalement \^etre d\'efinie des fa\c{c}ons suivantes 
(voir les prochains chapitres pour comprendre): 
\vspace*{2mm}

\begin{itemize}
\item[i)]
c'est la seule fonction continue qui \underline{transforme un produit en} 
\underline{somme}, $\ln(x\,y)=\ln(x)+ \ln(y)$, et qui vaut $1$ en $x=e$;
\vspace*{2mm}

\item[ii)]
c'est la seule \underline{primitive} de la fonction\quad 
$x\mapsto \dfrac{1}{x}$\quad qui vaut $0$ en $x=1$. 
\end{itemize}
\vspace*{2mm}

On a\quad \framebox{\ $D_{\ln}=]0,\infty[$\ }\quad et\quad 
\framebox{\ $I_{\ln}= \R$\ }. 
\end{itemize}

\end{frame}

%%%%%%%%%%%%%%%%%%%%%%%%%%%%%%%%%%%%%%%%%%%%%%%%%%

\begin{frame}
\frametitle{\bf Fonctions hyperboliques}
\medskip 

\begin{itemize}
\parbox{10cm}{
\bitem
{\bf cosinus hyperbolique}\ 
\framebox{\ $f(x)=\ch x = \cosh x = \dfrac{e^x+e^{-x}}{2}$\ }\quad avec\quad 
\framebox{\ $D_{\ch}\!=\!\R$\ }\quad et\quad 
\framebox{\ $I_{\ch}\!=\![1,\infty[$\ }
}
\vspace*{3mm}

\bitem
{\bf sinus hyperbolique}\ 
\framebox{\ $f(x)=\sh x= \sinh x = \dfrac{e^x-e^{-x}}{2}$\ }\quad avec\quad 
\framebox{\ $D_{\sh} \!=\!\R$\ }\quad et\quad \framebox{\ $I_{\sh}\!=\!\R$\ }
\vspace*{2mm}

\pause
\parbox[t]{7cm}{
On a
\vspace*{-2mm}
$$
\mbox{\framebox{\ $\ch^2x-\sh^2x=1$\ } pour tout $x\in\R$,}
$$ 
donc $(\ch x,\sh x)$ sont les coordonn\'ees des points $P$ qui se trouvent 
sur la branche droite de l'hyperbole d'\'equation $X^2-Y^2=1$} 

\quad\raisebox{-\height}{
\begin{picture}(60,60)(-15,-35)
\setlength{\unitlength}{1.5pt}
\put(-10,0){\vector(1,0){40}}
\put(28,-5){\scriptsize{$\vi$}}
\put(0,-25){\vector(0,1){55}}
\put(-5,28){\scriptsize{$\vj$}}
\qbezier(25,25)(-8,0)(25,-25)
%
\put(14,14){\circle*{2}}
\put(12,17){\scriptsize{$P$}}
\multiput(14,0)(0,4){4}{\line(0,1){2}}
\multiput(0,14)(4,0){4}{\line(1,0){2}}
%
\linethickness{0.5mm}
\put(0,0){\line(1,0){14}}
\put(2,-5){\scriptsize{$\ch x$}}
\put(0,0){\line(0,1){14}}
\put(-12,8){\scriptsize{$\sh x$}}
\end{picture}
}
\vspace*{3mm}

\pause
\bitem
{\bf tangente hyperbolique}\ 

\parbox{11cm}{
\framebox{\ $f(x)=\th x=\tanh x=\dfrac{\sh x}{\ch x}$\ }\quad 
avec\quad 
\framebox{\ $D_{\th}\!=\!\R$}\quad et\quad \framebox{\ $I_{\tan}\!=\!]\!-1,1[$\ }
}
\end{itemize}

\end{frame}

%\begin{frame}
%\frametitle{\bf Formulaire sur les fonctions hyperboliques}

%\centering{Voir livret de cours}

%\end{frame}
%%%%%%%%%%%%%%%%%%%%%%%%%%%%%%%%%%%%%%%%%%%%%%%%%%
%%%%%%%%%%%%%%%%%%%%%%%%%%%%%%%%%%%%%%%%%%%%%%%%%%

\subsection{Graphes} 

\begin{frame}[plain]
\only<1>{\addtocounter{framenumber}{-1}}
\frametitle{\bf 2. Graphe de fonctions} 
\medskip 

{\bf D\'efinition:} 
Le {\bf graphe} d'une fonction $f$ est l'ensemble des points du plan
\begin{align*}
\Gamma_f & = \Big\{ (x,y)\in \R^2\ |\ x\in D_f,\ y=f(x)\ \Big\} \\ 
& = \Big\{ \big(x,f(x)\big)\in \R^2\ |\ x\in D_f\ \Big\}
\subset \R^2
\end{align*}

$$
\grapheF
$$

\pause
En regardant le graphe d'une fonction on peut d\'eduire quel est son domaine, 
son image et ses propri\'et\'es importantes. 
\vspace*{3mm}

\pause
Le graphe des fonctions usuelles est \`a conna\^{\i}tre par c{\oe}ur.  

\end{frame}

%%%%%%%%%%%%%%%%%%%%%%%%%%%%%%%%%%%%%%%%%%%

\begin{frame}
\frametitle{\bf Graphes \`a conna\^{\i}tre !} 
\medskip 

\grapheX \quad \grapheXX \quad \grapheXXX %\quad \grapheXXXX

\grapheXinv \quad \grapheXXinv \quad \grapheXXXinv %\quad \grapheXXXXinv 

\grapheXsqrtII \quad \grapheXsqrtIII \quad \grapheXsqrtIV %\quad \grapheXsqrtV

\end{frame}

%%%%%%%%%%%%%%%%%%%%%%%%%%%%%%%%%%%%%%%%%%%

\begin{frame}[plain]
\only<1>{\addtocounter{framenumber}{-1}}
\frametitle{\bf D'autres graphes \`a conna\^{\i}tre !} 
\medskip 

\grapheXVA \quad \grapheXXXVA \quad \grapheXinvVA \quad %\grapheXsqrtIIIVA 

\grapheSIN \quad \grapheCOS \quad \grapheTAN %\quad \grapheCOT

\grapheASIN \quad \grapheACOS \quad \grapheATAN %\quad \grapheACOT

\end{frame}

%%%%%%%%%%%%%%%%%%%%%%%%%%%%%%%%%%%%%%%%%%%

\begin{frame}
\frametitle{\bf D'autres encore... ouf !} 
\medskip 

\qquad \qquad \grapheEXP \qquad \grapheEXPinv 

\qquad \qquad \grapheLN \qquad \grapheLNinv

\grapheSINH \qquad \grapheCOSH \qquad \grapheTANH

\end{frame}

%%%%%%%%%%%%%%%%%%%%%%%%%%%%%%%%%%%%%%%%%%%%%%%%%%
%%%%%%%%%%%%%%%%%%%%%%%%%%%%%%%%%%%%%%%%%%%%%%%%%%

\subsection{Croissance} 

\begin{frame}[plain]
\only<1>{\addtocounter{framenumber}{-1}}
\frametitle{\bf 3. Fonctions croissantes, d\'ecroissantes, monotones}
\medskip 

{\small 
La premi\`ere propri\'et\'e qu'on voit sur le graphe est la 
\underline{croissance}.}
\vspace*{2mm}

{\bf D\'efinition:} 
Soit $f:\R\rightarrow\R$ une fonction et $D\subset D_f$. On dit que: 
\vspace*{1mm}

\begin{itemize}
\bitem
\parbox[t]{7.5cm}{
$f$ est {\bf (strictement) croissante sur $D$} si\\ 
\framebox{\ $f(x)<f(y)$\ }\ pour tout $x,y\in D$ tels que $x<y$. 
}
\quad
\raisebox{-.7\height}{\croissant}  
\vspace*{1mm}

\bitem
\parbox[t]{7.5cm}{
$f$ est {\bf (strictement) d\'ecroissante sur $D$} si\\ 
\framebox{\  $f(x)\!>\!f(y)$\ }\ pour tout $x,y\in D$ tels que $x<y$.
}
\quad
\raisebox{-.7\height}{\decroissant}  
\vspace*{1mm}

\bitem
\parbox[t]{7.5cm}{
$f$ est {\bf constante sur $D$} si\ \framebox{\  $f(x)=f(y)$\ }\\ 
pour tout $x,y\in D$.
}
\quad
\raisebox{-.7\height}{\constant}  
\end{itemize} 
\vspace*{1mm}

\pause
{\small
Si on n'indique pas l'ensemble $D$, on sous-entend qu'on parle de tout le 
domaine de d\'efinition $D_f$.
} 
\vspace*{2mm}

\pause
\begin{itemize}
\bitem
$f$ est {\bf (strictement) monotone} si elle est partout croissante 
ou partout d\'ecroissante sur $D_f$.
\end{itemize} 
\vspace*{2mm}

\pause
{\small
L'appellatif ``strictement'' peut \^etre remplac\'e par ``largement'' 
si on consid\`ere des in\'egalit\'es larges $\leq$ et $\geq$. 

S'il est sous-entendu on consid\`ere les in\'egalit\'es strictes $<$ et $>$.
}

\end{frame}

%%%%%%%%%%%%%%%%%%%%%%%%%%%%%%%%%%%%%%%%%%%%%%%%%%

\begin{frame}
\frametitle{\bf Exemple de fonctions monotones}
\medskip 

{\small 
{\bf Exemples:}
\begin{itemize}
\bitem
Les polyn\^omes $x^n$ sont monotones croissants seulement si $n$ 
est impair. 
\vspace*{1mm}

Si $n$ est pair, ils sont d\'ecroissants pour $x<0$ et croissants pour $x>0$.  
\vspace*{1mm}

\parbox{10cm}{
\bitem
Les fractions $\frac{1}{x^n}$ sont monotones d\'ecroissantes seulement 
si $n$ est impair. 
\vspace*{1mm}

Si $n$ est pair, elles sont croissantes pour $x<0$ et d\'ecroissantes 
pour $x>0$.}
\vspace*{1mm}

\bitem
Les racines $\sqrt[k]{x}$ sont monotones croissantes. 
\vspace*{1mm}

\bitem
Les fonctions circulaires $\sin x$ et $\cos x$ ne sont pas monotones 
(elles sont {\bf oscillantes}).
La tangente $\tan x$ est monotone croissante. 
\vspace*{1mm}

\bitem
Les fonctions $\arcsin x$ et $\arctan x$ sont monotones croissantes, 
alors que $\arccos x$ est monotone d\'ecroissante. 
\vspace*{1mm}

\bitem
L'exponentiel $e^x$ et le logarithme $\ln x$ sont monotones croissants. 
\vspace*{1mm}

\bitem
Les fonctions hyperboliques $\sinh x$ et $\tanh x$ sont monotones 
croissantes, alors que $\cosh x$ est d\'ecroissant pour $x<0$ et 
croissant pour $x>0$. 
\end{itemize} 

}

\end{frame}

%%%%%%%%%%%%%%%%%%%%%%%%%%%%%%%%%%%%%%%%%%%%%%%%%%

\begin{frame}[plain]
\only<1>{\addtocounter{framenumber}{-1}}
\frametitle{\bf Fonctions convexes et concaves}
\medskip 

{\small 
La deuxi\`eme propri\'et\'e qu'on voit sur le graphe est la 
\underline{convexit\'e}.}
\vspace*{2mm}

{\bf D\'efinition:} 
Soit $f:\R\rightarrow\R$ une fonction et $D\subset D_f$. On dit que: 
\vspace*{1mm}

\begin{itemize}
\bitem
\parbox[t]{6cm}{
$f$ est {\bf convexe sur $D$} si elle a la forme 
}
\quad
\raisebox{-.5\height}{\convexe}  
\vspace*{1mm}

\bitem
\parbox[t]{6cm}{
$f$ est {\bf concave sur $D$} si elle a la forme
}
\quad
\raisebox{-.5\height}{\concave} 
\vspace*{1mm}

\bitem
\parbox[t]{6cm}{
$f$ est {\bf plate sur $D$} si elle est constante
}
\quad
\raisebox{-.5\height}{\constant}  
\end{itemize} 
\vspace*{1mm}

\pause
{\small
Si on n'indique pas l'ensemble $D$, on sous-entend qu'on parle de tout le 
domaine de d\'efinition $D_f$.
} 
\vspace*{2mm}

\pause
{\small 
{\bf Exemples:}
\begin{itemize}
\bitem
Les polyn\^omes $x^n$ et les fractions $\frac{1}{x^n}$ sont convexes si $n$ 
est pair. 
\vspace*{1mm}

Si $n$ est impair, ils sont concaves pour $x<0$ et convexes pour $x>0$.  
\vspace*{1mm}

\bitem
Les racines $\sqrt[k]{x}$ sont concaves. 
\vspace*{1mm}

\bitem
L'exponentiel $e^x$ est convexe. Le logarithme $\ln x$ est concave. 
\end{itemize} 
}

\end{frame}

%%%%%%%%%%%%%%%%%%%%%%%%%%%%%%%%%%%%%%%%%%%%%%%%%%

\begin{frame}
\frametitle{\bf Fonctions paires, impaires et p\'eriodiques}
\medskip 

{\small 
La troisi\`eme propri\'et\'e qu'on voit sur le graphe est la 
\underline{sym\'etrie}.}
\vspace*{2mm}

{\bf D\'efinition:} 
Soit $f:\R\rightarrow\R$ une fonction. On dit que: 
\vspace*{1mm}

\begin{itemize}
\bitem
\parbox[t]{6cm}{
$f$ est {\bf paire} si\ \framebox{\ $f(-x)=f(x)$\ }\ pour tout 
$x\!\in\! D_f$ (\underline{sym\'etrie axiale}).}
\quad
\raisebox{-.8\height}{\pair}
\vspace*{1mm}

\bitem
\parbox[t]{6cm}{
$f$ est {\bf impaire} si\ \framebox{\ $f(-x)=-f(x)$\ }\ pour tout 
$x\!\in\! D_f$ (\underline{sym\'etrie centrale}).}
\quad
\raisebox{-.8\height}{\impair} 
\vspace*{1mm}

\bitem
\parbox[t]{5.5cm}{
$f$ est {\bf p\'eriodique} de {\bf p\'eriode $p$} si\ 
\framebox{\ $f(x+p)=f(x)$\ }\ pour tout $x\!\in\! D_f$
(\underline{sym\'etrie par translation}).}
\quad
\raisebox{-.8\height}{\periodique} 
\end{itemize} 
\vspace*{0.5mm}

\pause
{\small 
{\bf Exemples:}
\begin{itemize}
\parbox{10cm}{
\bitem
Les polyn\^omes $x^n$ et les fractions $\frac{1}{x^n}$ sont des fonctions 
paires si $n$ est pair, et des fonctions impaires si $n$ est impair.}  
%\vspace*{1mm}
\parbox{10cm}{
\bitem
Les fonctions $\sin x$ et $\tan x$ sont impaires, $\cos x$ est paire. 
Toutes les trois sont p\'eriodiques: $\sin x$ et $\cos x$ de p\'eriode $2\pi$, 
$\tan x$ de p\'eriode $\pi$.}
\end{itemize} 

}

\end{frame}

%%%%%%%%%%%%%%%%%%%%%%%%%%%%%%%%%%%%%%%%%%%%%%%%%%

\begin{frame}[plain]
\only<1>{\addtocounter{framenumber}{-1}}
\frametitle{\bf Exercice}
\medskip 

{\bf Exercice:}
{\em 
Pour les fonctions suivantes, dessiner le graphe, pr\'eciser le domaine de 
d\'efinition et si elles sont monotones (croissantes ou d\'ecroissantes), 
paires ou impaires et p\'eriodiques.
}
\vspace*{2mm}

\begin{itemize}
\bitem
$f(x)=2\ln x +1$
\end{itemize}
\vspace*{2mm}

\pause
{\bf R\'eponse:}
Le graphe de $f(x)= 2\ln x +1$ se trouve en dilatant par $2$ le graphe de 
$x\mapsto \ln x$ et en d\'ecalant tout de $+1$: 
$$
\raisebox{-\height}{\exoLNA} \qquad\qquad 
\raisebox{-\height}{\exoLNB}
$$
\pause
Le domaine de $f$ est\quad $D_f=\{ x\in\R\ |\ x>0\} = ]0,\infty[$. 
\vspace*{1mm}

La fonction $f$ est monotone croissante, ni paire ni impaire. 

\end{frame}

%%%%%%%%%%%%%%%%%%%%%%%%%%%%%%%%%%%%%%%%%%%%%%%%%%

\begin{frame}[plain]
\frametitle{\bf Exercice (suite)}
\medskip 

\begin{itemize}
\bitem
$u(\theta)=\cos (2\theta)-1$
\end{itemize}
\vspace*{2mm}

\pause
{\bf R\'eponse:}
Le graphe de $u(\theta)= \cos (2\theta)-1$ se trouve en d\'ecalant de $-1$ 
le graphe de la fonction $f(x)= \cos x$ o\`u $x=2\theta$: 
$$
\raisebox{-\height}{\exoCOSA} \qquad\qquad 
\raisebox{-\height}{\exoCOSB}
$$
\pause
Le domaine de $u$ est\quad $D_u=\{ \theta\in\R\ |\ 2\theta\in\R \} = \R$. 
\vspace*{1mm}

La fonction $u$ n'est pas monotone, et elle est paire. 
\vspace*{1mm}

Elle est clairement p\'eriodique de p\'eriode $\pi$: 
\begin{align*}
u(\theta+\pi) & = \cos\big(2(\theta+\pi)\big)-1 
= \cos(2\theta+2\pi)-1 \\ 
& = \cos(2\theta)-1 = u(\theta).
\end{align*}   

\end{frame}

\begin{frame}[plain]
\only<1>{\addtocounter{framenumber}{-1}}
\frametitle{\bf Exercice (suite)}
\medskip 

\begin{itemize}
\bitem
$z(t)=-\sqrt{t-1}$
\end{itemize}
\vspace*{2mm}

\pause
{\bf R\'eponse:}
Le graphe de $z(t)= -\sqrt{t-1}$ se trouve par \'etapes: 
on dessine la fonction $\sqrt{t}$, on d\'ecale la variable independante 
de $t$ \`a $t-1$ en bougeant l'axe vertical de $-1$ en horizontal, 
enfin on prend son oppos\'e $-\sqrt{t-1}$.
$$
\raisebox{-\height}{\exoSQRTA} \qquad
\raisebox{-\height}{\exoSQRTB} \qquad
\raisebox{-\height}{\exoSQRTC} 
$$
\pause
Le domaine de la fonction $z$ est\quad 
$D_z=\{ t\in\R\ |\ t-1\geq 0 \} = [1,\infty[$. 
\vspace*{1mm}

\parbox{10cm}{
La fonction $z$ est monotone d\'ecroissante, elle n'est ni paire ni impaire.}

\end{frame}

%%%%%%%%%%%%%%%%%%%%%%%%%%%%%%%%%%%%%%%%%%%%%%%%%%
%%%%%%%%%%%%%%%%%%%%%%%%%%%%%%%%%%%%%%%%%%%%%%%%%%

\subsection{Op\'erations} 

\begin{frame}
\frametitle{\bf 4. Op\'erations entre fonctions}
\medskip 

{\bf D\'efinition:} 
Soient $f,g:\R\rightarrow\R$ deux fonctions, et $t\in\R$. 
\vspace*{1mm}

\begin{itemize}
\bitem
{\bf addition}:\ \framebox{\ $(f+g)(x)=f(x)+g(x)$\ } avec domaine 
\vspace*{-1mm}
$$
D_{f+g} =\{x\in\R\ |\ \mbox{$x\in D_f$ et $x\in D_g$} \} = D_f\cap D_g
\vspace*{-1mm}
$$

{\bf z\'ero}:\ \framebox{\ $0(x)=0$\ } avec $D_0=\R$
\vspace*{1mm}

{\bf oppos\'ee}: \framebox{\ $(-f)(x)=-f(x)$\ } avec $D_{-f}=D_f$
\vspace*{1mm}

\pause
\bitem
{\bf produit par un scalaire}:\ \framebox{\ $(t\,f)(x)=t\,f(x)$\ } avec 
$D_{t\,f}=D_f$
\vspace*{1mm}

\pause
\bitem
{\bf multiplication}:\ \framebox{\ $(f\,g)(x)=f(x)g(x)$\ } avec 
$D_{f\,g}=D_f\cap D_g$
\vspace*{1mm}

{\bf unit\'e}:\ \framebox{\ $1(x)=1$\ } avec $D_1=\R$
\vspace*{0.5mm}

{\bf inverse}: \framebox{\ $\left(\frac{1}{f}\right)(x)=\frac{1}{f(x)}$\ } 
avec $D_{1/f}=\{x\in D_f\ |\ f(x)\neq 0 \}$
\end{itemize}
\vspace*{0.5mm}

\pause
{\small 
{\bf Exemples:}
\begin{itemize}
\item
$h(x)=x^2+\sin x$ est la somme de $f(x)=x^2$ et $g(x)=\sin x$

\item
$k(x)=10(x^2+\sin x)$ est le produit de $h(x)$ par le scalaire $10$ 

\item
$H(x)=\frac{x^2}{\sin x}$ est le produit de $f(x)$ par l'inverse de $g(x)$
\end{itemize}
}

\end{frame}

%%%%%%%%%%%%%%%%%%%%%%%%%%%%%%%%%%%%%%%%%%%%%%%%%%

\begin{frame}[plain]
\only<1>{\addtocounter{framenumber}{-1}}
\frametitle{\bf Propri\'et\'es des op\'erations}
\medskip 

{\bf Proposition:}
\begin{itemize}
\bitem
Les op\'erations entre fonctions ont les m\^emes \underline{propri\'et\'es} 
que leurs analogues entre nombres r\'eels {\small (associative, commutative, 
distributive)}. 

\bitem
En particulier, l'ensemble des fonctions est un \underline{espace vectoriel} 
(de dimension infinie) avec l'addition et le produit par un scalaire. 
\end{itemize}
\vspace*{2mm}

\pause
{\small 
{\bf Exemple:}
Si $f(x)=x^2$, $g(x)=\sin x$, $h(x)=\cos x$ et $t=10$, l'\'egalit\'e
$$
x^2(10\,\sin x+\cos x) = x^2\cos x+10\,x^2\sin x \qquad\mbox{(pour tout $x$)}
$$
s'exprime en terme de fonctions comme 
$$
f\,(t\,g+h) = f\,h + t\,f\,g
$$ 
et repose sur la propri\'et\'e \underline{commutative} de l'addition et du 
produit par un scalaire et sur la propriet\'e \underline{distributive} 
de la multiplication par rapport \`a l'addition.}
\vspace*{3mm}

\pause
{\bf Note:} Un espace vectoriel qui a en plus une \underline{multiplication} 
s'appelle {\bf alg\`ebre}.

\end{frame}

%%%%%%%%%%%%%%%%%%%%%%%%%%%%%%%%%%%%%%%%%%%%%%%%%%

\begin{frame}
\frametitle{\bf Composition de fonctions}
\medskip 

{\small 
La \underline{composition} de fonctions est une op\'eration qui n'a pas 
d'analogue dans les nombres r\'eels. }
\vspace*{1mm}

\pause
{\bf D\'efinition:} 
La {\bf compos\'ee} de deux fonctions $x\mapsto f(x)$ et $y\mapsto g(y)$ 
est la fonction $g\circ f:\R\rightarrow\R$ d\'efinie par
$$
\mbox{\framebox{\ $(g\circ f)(x)=g\big(f(x)\big)$\ }} 
$$
avec domaine\ \framebox{\ $D_{g\circ f}=\big\{ x\in D_f\ |\ f(x)\in D_g \big\}$\ }.
\vspace*{1mm}

\pause
La composition peut \^etre vue comme l'\underline{enchainement} 
des fonctions l'une apr\`es l'autre: 
$$
\begin{picture}(80,30)(10,-10)
\setlength{\unitlength}{1.5pt}
\put(0,10){$x$}
\put(10,12){\vector(1,0){13}}
\put(27,10){$f(x)$}
\put(43,12){\vector(1,0){13}}
\put(60,10){$g\big(f(x)\big)$}
\qbezier(3,7)(34,-3)(65,5)
\put(65,5){\vector(2,1){3}}
\put(25,-5){$(g\circ f)(x)$}
\end{picture}
$$
ou \'egalement comme la \underline{substitution} de la variable $y$, 
dans $g(y)$, par la valeur $y=f(x)$.   
\vspace*{1mm}

\pause
{\small 
{\bf Exemple:}\quad
Si $f(x)=x^2$ et $g(y)=\sin y$, on pose $y=x^2$ et on a: 
$$
(g\circ f)(x) = g\big(f(x)\big) = g(y)\Big|_{y=f(x)} 
= \sin y\Big|_{y=x^2} = \sin(x^2). 
$$
}

\end{frame}

%%%%%%%%%%%%%%%%%%%%%%%%%%%%%%%%%%%%%%%%%%%%%%%%%%

\begin{frame}[plain]
\only<1>{\addtocounter{framenumber}{-1}}
\frametitle{\bf Propri\'et\'es de la composition}
\medskip 

{\bf Propri\'et\'es:}
\begin{itemize}
\bitem 
La composition est \underline{associative}:\ 
\framebox{\ $\big(h\circ g\big) \circ f = h\circ \big(g\circ f\big)$\ } \\ 
mais elle \underline{n'est pas commutative}:\ 
\framebox{\ $g \circ f \neq f\circ g$\ } en g\'en\'eral. 
\end{itemize}
%\vspace*{1mm}

\pause
{\small 
{\bf Exemple:}\quad
Si\quad $f(x)=x^2$,\quad $g(y)=\sin y$\quad et\quad $h(z)=\ln z$,\quad on a
\begin{align*}
(h\circ g)(y) = \ln(\sin y) & \quad\mbox{donc}\quad 
\big((h\circ g) \circ f\big)(x) = \ln(\sin(x^2)) \\ 
(g\circ f)(x) = \sin(x^2) & \quad\mbox{donc}\quad 
\big(h\circ (g\circ f)\big)(x) = \ln(\sin(x^2))
\vspace*{-2mm}
\end{align*}
et 
\vspace*{-2mm}
$$
(g\circ f)(x) = \sin(x^2) \quad\mbox{mais}\quad 
(f\circ g)(y) = (\sin y)^2.
\vspace*{-2mm}
$$
}

\pause
\begin{itemize}
\bitem 
La fonction {\bf identit\'e}\ \framebox{\ $\id(x)=x$\ }, avec domaine 
$D_{\id}=\R$, est une \underline{unit\'e} pour la composition:\ 
\framebox{\ $f\circ\id = \id\circ f= f$\ }. 
\end{itemize}
\vspace*{0.5mm}

\pause
{\small 
{\bf Remarque:}\quad
Pour une fonction $f$, l'\underline{inverse} $\displaystyle\frac{1}{f}$ est d\'efinie 
\`a partir de la \underline{multiplication} et de l'\underline{unit\'e} $1$ 
de telle sorte que\ \framebox{\ $f\displaystyle\frac{1}{f}=1$\ }\ et\
\framebox{\ $\displaystyle\frac{1}{f} f=1$\ }. 

L'analogue pour la \underline{composition} et l'\underline{identit\'e} 
est une fonction $f^{-1}$ telle que\ \framebox{\ $f^{-1}\circ f=\id$\ }\ et\
\framebox{\ $f\circ f^{-1}=\id$\ }, c'est la \underline{r\'eciproque} de $f$. 
} 

\end{frame}

%%%%%%%%%%%%%%%%%%%%%%%%%%%%%%%%%%%%%%%%%%%%%%%%%%

\subsection{R\'eciproques} 

\begin{frame}
\frametitle{\bf 5. Fonctions r\'eciproques}
\medskip 

{\bf D\'efinition:}\quad La {\bf r\'eciproque} d'une fonction $x\mapsto f(x)$ 
est la fonction $y\mapsto f^{-1}(y)$ telle que 
$$
\mbox{
\framebox{\ $f^{-1}\circ f=\id$\ }\quad et \quad 
\framebox{\ $f\circ f^{-1}=\id$\ }
}
$$
c'est-\`a-dire telle que 
$$
\mbox{
\framebox{\ \parbox{2.5cm}{$f^{-1}(f(x))=x$ \\ pour tout $x\in D_f$}\ }
\quad et \quad 
\framebox{\ \parbox{2.5cm}{$f(f^{-1}(y))=y$ \\ pour tout $y\in I_f$}\ }
}
$$
\pause
ce qui peut \^etre r\'esum\'e en une seule assertion: 
$$
\mbox{
\framebox{\ $f^{-1}(y)=x \quad\Longleftrightarrow\quad y=f(x)$\ }
}
$$
\pause
Ceci implique que\quad 
\framebox{\ $D_{f^{-1}}=I_f$\ }\quad et \quad 
\framebox{\ $I_{f^{-1}}=D_f$\ }. 
\vspace*{2mm}

\pause
En conclusion, on peut visualiser la r\'eciproque comme ceci:
$$
\begin{picture}(120,30)(-55,-10)
\setlength{\unitlength}{1.5pt}
\put(-57,10){$D_f=I_{f^{-1}}$}
\put(-57,0){$x=f^{-1}(y)$}
\qbezier(-20,3)(0,8)(20,3)
\put(20,3){\vector(2,-1){2}}
\put(0,7){$f$}
\put(25,0){$y=f(x)$}
\put(25,10){$I_f=D_{f^{-1}}$}
\qbezier(-20,-2)(0,-7)(20,-2)
\put(-20,-2){\vector(-2,1){2}}
\put(-3,-11){$f^{-1}$}
\end{picture}
$$

\end{frame}

%%%%%%%%%%%%%%%%%%%%%%%%%%%%%%%%%%%%%%%%%%%%%%%%%%

\begin{frame}[plain]
\only<1>{\addtocounter{framenumber}{-1}}
\frametitle{\bf Exemples de r\'eciproques}
\medskip 

{\small 
{\bf Exemples:}
\begin{itemize}
\bitem
La r\'eciproque de l'exponentiel $f(x)=e^x$ est le logarithme 
$f^{-1}(y)=\ln y$, car 
$$
f^{-1}(f(x))=\ln(e^x)=x \quad\mbox{et}\quad 
f(f^{-1}(y))=e^{\ln y}=y, 
$$
c'est-\`a-dire\quad $e^x=y\quad\Longleftrightarrow\quad x=\ln y$. 
\vspace*{2mm}

\pause
\bitem
La r\'eciproque de la fonction $g(x)=x^3+1$ se trouve en posant $x^3+1=y$ 
et en calculant $x=g^{-1}(y)$ comme fonction de $y$: 
$$
y=x^3+1 \quad\Longleftrightarrow\quad y-1=x^3 \quad\Longleftrightarrow\quad 
x=\sqrt[3]{y-1}
$$
donc $g^{-1}(y)=\sqrt[3]{y-1}$. 
\vspace*{2mm}

\pause
\bitem
La r\'eciproque de la fonction $h(x)=\dfrac{5}{x^3+1}$ se trouve en posant 
$\dfrac{5}{x^3+1}=y$ et en calculant $x=h^{-1}(y)$ comme fonction de $y$: 
$$
y=\dfrac{5}{x^3+1} \quad\Longleftrightarrow\quad x^3+1=\dfrac{5}{y} 
\quad\Longleftrightarrow\quad 
x=\sqrt[3]{\dfrac{5}{y}-1}
$$
donc $h^{-1}(y)=\sqrt[3]{\dfrac{5}{y}-1}$. 
\end{itemize}

}

\end{frame}

%%%%%%%%%%%%%%%%%%%%%%%%%%%%%%%%%%%%%%%%%%%%%%%%%%

\begin{frame}
\frametitle{\bf Propri\'et\'es des r\'eciproques}
\medskip 

{\bf \th\'eor\`eme:}\quad 
La r\'eciproque d'une fonction $f$ \underline{existe} si et seulement si 
$f$ est \underline{strictement monotone}. 
\vspace*{1mm}

\pause
{\small 
{\bf Id\'ee:}\quad 
En effet, si $f$ n'est pas strictement monotone, il existe deux points 
distincts $x_1$ et $x_2$ qui donnent la m\^eme valeur $y=f(x_1)=f(x_2)$. 

Dans ce cas, comment va-t-on d\'efinir la r\'eciproque $f^{-1}$ au point $y$, 
$f^{-1}(y)=x_1$ ou bien $f^{-1}(y)=x_2$? Ce choix est impossible.}
\vspace*{1mm}

\pause
{\bf Propri\'et\'es:}
\begin{itemize}
\bitem
Si $f$ est strictement monotone et on note $\Gamma_f$ son graphe, 
la r\'eciproque $f^{-1}$ est aussi strictement monotone et son graphe 
est l'image miroir de $\Gamma_f$ par rapport \`a la droite $y=x$. 
$$
\grapheFF \qquad \grapheFFINV
$$

\pause
\bitem
La r\'eciproque de la r\'eciproque de $f$ est $f$:\ 
\framebox{\ $\big(f^{-1}\big)^{-1}=f$\ }.
\end{itemize}

\end{frame}

%%%%%%%%%%%%%%%%%%%%%%%%%%%%%%%%%%%%%%%%%%%%%%%%%%

\begin{frame}[plain]
\only<1>{\addtocounter{framenumber}{-1}}
\frametitle{\bf R\'eciproque des fonctions non monotones}
\medskip 

{\bf Probl\`eme:}
\begin{itemize}
\bitem
Les \underline{polyn\^omes} $x^n$ de \underline{puissance impaire} et 
l'\underline{exponentiel} $e^x$ sont monotones et admettent 
une r\'eciproque, \`a savoir respectivement les \underline{racines} $\sqrt[n]{x}$ 
d'\underline{ordre impair} et le \underline{logarithme} $\ln{x}$: 
\begin{center}
\framebox{\ $x=\sqrt[n]{y} \quad\Longleftrightarrow\quad x^n=y$\ } 
\\ 
\framebox{\ $x=\ln y \quad\Longleftrightarrow\quad e^x=y$\ }
\end{center}
 \vspace*{1mm}

\pause
\bitem
Mais les \underline{polyn\^omes} $x^n$ de \underline{puissance paire} et  
les \underline{fonctions circulaires} $\sin x$, $\cos x$ et $\tan x$ 
ne sont pas monotones et n'admettent donc pas de r\'eciproque! 
Que faire?
\end{itemize}
\vspace*{1mm}

\pause
{\bf Astuce:}\quad 
Si une fonction $f$ n'est pas monotone, on peut restreindre 
son domaine de d\'efinition \`a un ensemble $D\subset D_f$ tel que 
\vspace*{-1mm}

\qquad
\parbox{9cm}{
\begin{itemize}
\item[i)]
$f$ soit monotone sur $D$,
\item[ii)]
$f(D)=I_f$. 
\end{itemize}}
\vspace*{-1mm}

Cette fonction ``\`a domaine restreint''\ 
\framebox{\ $f:D\subset D_f\rightarrow I_f$\ }\  
admet bien une r\'eciproque ``\`a image restreinte'':
$$
\mbox{\framebox{\ $f^{-1}:I_f\rightarrow D\subset D_f$\ }}. 
$$

\end{frame}

%%%%%%%%%%%%%%%%%%%%%%%%%%%%%%%%%%%%%%%%%%%%%%%%%%

\begin{frame}
\frametitle{\bf Exemples de r\'eciproques ``restreintes''}
\medskip 

{\small 
{\bf Exemples:}
\begin{itemize}
\bitem
Les \underline{polyn\^omes} $x^n$ de \underline{puissance paire} 
restreints \`a l'ensemble $[0,\infty[ \subset \R$ ont comme 
r\'eciproque les \underline{racines} $\sqrt[n]{x}$ d'\underline{ordre pair}:
$$
\mbox{\framebox{\ $x=\sqrt[n]{y} \qquad\Longleftrightarrow\qquad 
x^n=y \quad \mbox{et}\quad x\geq 0 $\ }}
$$
Ex.:\quad 
\raisebox{-.5\height}{\grapheXXTOT \quad \grapheXXRED \quad \grapheSQRT}

\pause
\bitem
Les \underline{fonctions circulaires} $\sin x$, $\cos x$ et $\tan x$, 
opportunement restreintes, ont comme r\'eciproque les 
\underline{fonctions arc} $\arcsin x$, $\arccos x$ et $\arctan x$:  
\begin{center}
\framebox{\ $x=\arcsin{y} \qquad\Longleftrightarrow\qquad 
\sin x =y \quad \mbox{et}\quad x\in [-\pi/2,\pi/2]$\ } 
\\ 
\framebox{\ $x=\arccos{y} \qquad\Longleftrightarrow\qquad 
\cos x =y \quad \mbox{et}\quad x\in [0,\pi]$\ }
\\  
\framebox{\ $x=\arctan{y} \qquad\Longleftrightarrow\qquad 
\tan x =y \quad \mbox{et}\quad x\in ]-\pi/2,\pi/2[$\ } 
\end{center}
Ex.:\quad 
\raisebox{-.5\height}{\grapheSINTOT \quad \grapheSINRED \quad \grapheASINRED}
\end{itemize}

}

\end{frame}

%%%%%%%%%%%%%%%%%%%%%%%%%%%%%%%%%%%%%%%%%%%%%%%%%%

\begin{frame}[plain]
\only<1>{\addtocounter{framenumber}{-1}}
\frametitle{\bf Exercice}
\medskip 

{\bf Exercice:} 
{\em Calculer la r\'eciproque des fonctions suivantes.}
\vspace*{2mm}

\begin{itemize}
\bitem
$f(x)=3x^2-5$, avec $x\geq 0$
\end{itemize}
\vspace*{2mm}

\pause
{\bf R\'eponse:}\vspace*{-5mm} 
$$
y=3x^2-5 \quad\Leftrightarrow\quad 
\frac{y+5}{3}=x^2 \quad\Leftrightarrow\quad 
x=\sqrt{\frac{y+5}{3}}
$$

\pause
\begin{itemize}
\bitem
$f(\theta)=\sqrt{\sin\theta+3}$, avec $-\pi/2\leq \theta\leq \pi/2$
\end{itemize}
\vspace*{2mm}

\pause
{\bf R\'eponse:} \vspace*{-2mm} 
$$
x=\sqrt{\sin\theta+3} \quad\Leftrightarrow\quad 
x^2-3=\sin\theta \quad\Leftrightarrow\quad 
\theta=\arcsin(x^2-3)
$$

\pause
\begin{itemize}
\bitem
$f(z)=3\arctan(e^z)$ 
\end{itemize}
\vspace*{3mm}

\pause
{\bf R\'eponse:}  \vspace*{-2mm} 
$$ 
t/3=\arctan(e^z) \quad\Leftrightarrow\quad 
\tan(t/3)=e^z \quad\Leftrightarrow\quad 
z=\ln(\tan(t/3))
$$
avec\quad $-\pi/2<t/3<\pi/2$\quad et \quad $\tan(t/3)>0$,\quad 
c'est-\`a-dire\quad $0<t/3<\pi/2$,\quad au final:\quad $0<t<3\pi/2$.

\end{frame}

%%%%%%%%%%%%%%%%%%%%%%%%%%%%%%%%%%%%%%%%%%%%%%%%%%

\begin{frame}
\end{frame}

%%%%%%%%%%%%%%%%%%%%%%%%%%%%%%%%%%%%%%%%%%%%%%%%%%

%\begin{frame}
%\end{frame}

%%%%%%%%%%%%%%%%%%%%%%%%%%%%%%%%%%%%%%%%%%%%%%%%%%

%\begin{frame}
%\end{frame}

%%%%%%%%%%%%%%%%%%%%%%%%%%%%%%%%%%%%%%%%%%%%%%%%%%%%%%%%%%%%%%%%%%%%%
%%%%%%%%%%%%%%%%%%%%%%%%%%%%%%%%%%%%%%%%%%%%%%%%%%%%%%%%%%%%%%%%%%%%%

\end{document}