
%
%  TMB-diapo.tex  2020-2021  A. Mikelic & L. Dupaigne
%
%%%%%%%%%%%%%%%%%%%%%%%%%%%%%%%%%%%%%%%%%%%%%%%%%%%%%%%%%%%%%%%%%%%%%%
%
% compiler avec pdflatex pour produire les slides avec pauses
% compiler [handout] pour produire un fichier sans pauses, à imprimer
%
%%%%%%%%%%%%%%%%%%%%%%%%%%%%%%%%%%%%%%%%%%%%%%%%%%%%%%%%%%%%%%%%%%%%

%\documentclass[10pt]{beamer}
\documentclass[8pt]{article} % to get handout of the presentation

%\usetheme[height=1cm,width=1.7cm]{}%{sidebar}	

%	
% packages 
%
\usepackage{amsfonts,amsmath,amssymb,amscd,amstext,mathabx,bbm}
\usepackage{textcomp,multicol,multirow,enumitem}
\usepackage{pict2e,graphicx}
\usepackage{mathrsfs}


\usepackage[utf8]{inputenc}
 \usepackage[francais]{babel}  % bof 

%\setlength{\textwidth}{17cm}
%\setlength{\oddsidemargin}{-1.05cm}
%\setlength{\evensidemargin}{-1.05cm}
%
% packages pour dessins avec tikzpicture
%
%\usepackage{pgf,tikz,pgfplots}
%\pgfplotsset{compat=1.15}
%\usepackage{mathrsfs}
%\usetikzlibrary{arrows}
\usepackage{tikz}
\usepackage{tikz-3dplot}

%
%   TMB-macros.sty  2015  
%
%%%%%%%%%%%%%%%%%%%%%%%%%%%################################%%%%%%%%%%%%%%%%%%%%%%%%%%%%%%%%%%%%%%%%%%%%%%


%\setlength{\unitlength}{.3cm}

%\renewcommand{\arraystretch}{1.5}

\renewcommand{\contentsname}{Programme du cours}

\newcommand{\bitem}{\item[$\bullet$]}
\newcommand{\litem}{\hspace{-.5cm}\item[$\bullet$]}
\setlist[itemize]{leftmargin=*}

\newcommand{\ds}{\displaystyle}

\newcommand{\N}{\mathbb N}
\newcommand{\Z}{\mathbb Z}
\newcommand{\Q}{\mathbb Q}
\newcommand{\R}{\mathbb R}
\newcommand{\C}{\mathbb C}

\newcommand{\lra}{\longrightarrow}

\newcommand{\Arg}{\mathrm{Arg}}
\renewcommand{\Re}{\mathrm{Re}}
\renewcommand{\Im}{\mathrm{Im}}

\newcommand{\dist}{\mathrm{dist}}
\newcommand{\Vect}{\mathrm{Vect}} 
\newcommand{\Lin}{\mathrm{Lin}}  
\newcommand{\Rot}{\mathrm{Rot}} 
% \newcommand{\Ref}{\mathrm{Ref}} 
\newcommand{\calF}{{\cal F}}
\newcommand{\calL}{{\cal L}}
\newcommand{\calM}{{\cal M}}
\newcommand{\calS}{{\cal S}}
\newcommand{\matun}{\mathbbm{1}}
\newcommand{\Aire}{\mathrm{Aire}}
\newcommand{\Vol}{\mathrm{Vol}} 
\newcommand{\id}{\mathrm{id}}

\newcommand{\vi}{\vec{\mbox{\it \i}}\,}
\newcommand{\vj}{\vec{\mbox{\it \j}}\,} 
\newcommand{\vk}{\vec{\mbox{\it k}}\,} 
\newcommand{\ve}{\vec{e}}
\newcommand{\vv}{\vec{v}}
\newcommand{\vu}{\vec{u}}
\newcommand{\vw}{\vec{w}}
\newcommand{\vx}{\vec{x}}
\newcommand{\vy}{\vec{y}}
\newcommand{\vXY}[2]{\left(\!\!\begin{array}{c} #1 \\ #2 \end{array}\!\!\right)}
\newcommand{\vXYZ}[3]{\left(\!\!\begin{array}{c} #1 \\ #2 \\ #3 \end{array}\!\!\right)}

\newcommand{\vA}{\overrightarrow{A}}
\newcommand{\vB}{\overrightarrow{B}}
\newcommand{\vE}{\overrightarrow{E}}
\newcommand{\vF}{\overrightarrow{F}}
\newcommand{\vG}{\overrightarrow{{\cal G}}}
\newcommand{\vV}{\overrightarrow{V}}
\newcommand{\vU}{\overrightarrow{U}}
\newcommand{\vdl}{\overrightarrow{d\ell}}
\newcommand{\vdS}{\overrightarrow{dS}}

\newcommand{\ch}{\mathrm{ch}\,} 
\newcommand{\sh}{\mathrm{sh}\,} 
\renewcommand{\th}{\mathrm{th}\,} 

\newcommand{\ddx}{\frac{d}{dx}}
\newcommand{\dd}[1]{\frac{d #1}{dx}}

%%%%%%%%%%%%%%%%%%%%%%%%%%%%%%%%%%%%%%%%%%%%%%%%%%%%%%%%%%%%%%%%%%%%%%%%

%\newenvironment{definition}{\medskip \noindent{\bf D\'efinition.}\ }
%{\medskip}
\newenvironment{theoreme}{\medskip \noindent{\bf Th\'eor\`eme.}\ }
{\medskip}
\newenvironment{theoreme-nom}[1]{\medskip \noindent{\bf Th\'eor\`eme #1.}\ }
{\medskip}
\newenvironment{proposition}{\medskip \noindent{\bf Proposition.}\ }
{\medskip}
\newenvironment{corollaire}{\medskip \noindent{\bf Corollaire.}\ }
{\medskip}
\newenvironment{lemme}{\medskip \noindent{\bf Lemme.}\ }
{\medskip}
\newenvironment{remarque}{\medskip \noindent{\bf Remarque.}\ }{\medskip}
\newenvironment{rappel}{\medskip \noindent{\bf Rappel.}\ }{\medskip}
\newenvironment{exercice}{\medskip \noindent{\bf Exercice.}\ }{\medskip}
\newenvironment{exercices}{\medskip \noindent{\bf Exercices.}\ }{\medskip}
%\newenvironment{exemple}{\medskip \begin{quote}\noindent{\bf Exemple.}\quad}%
%     {\end{quote}\medskip}
%\newenvironment{exemples}{\medskip \begin{quote}\noindent{\bf Exemples.}\quad}%
%     {\end{quote}\medskip}
\newenvironment{exemple}{\medskip \noindent{\bf Exemple.}\ }{\medskip}
\newenvironment{exemples}{\medskip \noindent{\bf Exemples.}\ }{\medskip}
\newenvironment{preuve}{\medskip \noindent{\bf Preuve.}\ }%
{\medskip}
\newenvironment{application}{\medskip \noindent{\bf Application.}\ }%
{\medskip}
\newenvironment{question}{\medskip \noindent{\bf Question.}\ }{\medskip}
\newenvironment{conclusion}{\medskip \noindent{\bf Conclusion.}\ }{\medskip}


%%%%%%%%%%%%%%%%%%%%%%%%###########################%%%%%%%%%%%%%%%%%%%%%%%%%%%%%%%%%%%%%%%%%%%%%%%









% \usepackage{TMB-dessins}

%
% macros
%
% \usepackage{TMB-macros}
%\usepackage[paperheight=11cm, paperwidth=8.5cm]{geometry}
%
% macros pour beamer 
%
\newcommand{\uv}{\mbox{\small $u$,$v$}}
\newcommand{\xy}{\mbox{\small $x$,$y$}}
\newcommand{\xxyy}{\mbox{\footnotesize $\sqrt{x^2\!+\!y^2}$}}

%\AtBeginSubsection[]
%{
%  \begin{frame}<beamer>
%    \frametitle{Plan}
%    \tableofcontents[currentsection,currentsubsection]
%  \end{frame}
%}



%%%%%%%%%%%%%%%%%%%%%%%%%%%%%%%%%%%%%%%%%%%%%%%%%%%%%%%%%%%%%%%%%%%%%
%%%%%%%%%%%%%%%%%%%%%%%%%%%%%%%%%%%%%%%%%%%%%%%%%%%%%%%%%%%%%%%%%%%%%

\begin{document}

% \newcommand{\fonctions_circulaires}{
\begin{tikzpicture}[scale = 1]%, domain=0:6.28 ]
    %% axis %% 
    \draw[->,>=stealth'] (-1.5,0) -> (1.5,0)  
        node[below right]{\scriptsize{$\vi$}};
    \draw[->,>=stealth'] (0,-1.5) -> (0,1.5) 
        node[above]{\scriptsize{$\vj$}} ;
    \draw[->,>=stealth'] (1,-0.75) -> (1,1.25) ;%tan axis<
    %% cercle unite  : 
    \draw (0,0) circle (1); 
    %% ticks : 
    \draw (0,0) circle (0.03)[fill=black] node[below left] {\scriptsize $O$};
    \draw (1,2pt) -- (1,-2pt) node[below right] {\scriptsize $1$};
 	\draw (2pt,1) -- (-2pt,1) node[above left] {\scriptsize $1$}; 
    %%  dot : 
    \draw (45:1) circle(0.03)[fill=black] 
        node[above]{\scriptsize $P$};
    \draw[densely dashed] (45:1) -- (0.707,0) circle(0.03)[fill=black] ;
    \draw[densely dashed] (45:1) -- (0,0.707) circle(0.03)[fill=black];
    \draw (1,1) circle(0.03)[fill=black];
    \draw (1,0) circle(0.03)[fill=black];
    \draw (0,0) -- (1,1);% node[ rotate = 45, midway, above left]{\scriptsize{$x$}}; 
    \draw[->,densely dotted] (0:0.3) arc[radius=0.3, start angle=0, end angle=45] 
        node[midway, right]{\scriptsize $x$};
    %% cos, sin, tan : 
    \draw[very thick] (0,0) -- (0,0.707) 
        node[rotate = 90, midway, above]{\scriptsize $\sin x$}; 
    \draw[very thick] (0,0) -- (0.707,0)    
        node[midway,below]{\scriptsize $\cos x$}; 
    \draw[very thick] (1,0) -- (1,1)
        node[rotate = 270, midway, above]{\scriptsize $\tan x$}; 
\end{tikzpicture}
% }



% \newcommand{\fonctions_hyperboliques}{
\begin{tikzpicture}[scale = 1]%, domain=0:6.28 ]
    %% axis %% 
    \draw[->,>=stealth'] (-1.5,0) -> (1.5,0)  
        node[below right]{\scriptsize{$\vi$}};
    \draw[->,>=stealth'] (0,-1.25) -> (0,1.5) 
        node[above]{\scriptsize{$\vj$}} ;
    %%  dot \sqrt(1.2^2 -1) = 0.663: 
    \draw (1.2,0.663) circle(0.03)[fill=black] 
        node[above]{\scriptsize $P$};
    \draw[densely dashed] (1.2,0.663) -- (1.2,0) circle(0.03)[fill=black] ;
    \draw[densely dashed] (1.2,0.663) -- (0,0.663) circle(0.03)[fill=black];
    % \draw (0,0) -- (1.2,0.663); 
    %% cosh, sinh : 
    \draw[very thick] (0,0) -- (0,0.663) 
        node[rotate = 90, midway, above]{\scriptsize $\sh x$}; 
    \draw[very thick] (0,0) -- (1.2,0)    
        node[midway,below]{\scriptsize $\ch x$}; 
    %% X^2 - Y^2 = 1
    \begin{scope}
        \draw plot[variable=\x,domain=1:2,samples=20,smooth] 
         ({\x},{(\x*\x - 1)^(1/2)}); %% (y, z, x)
        \draw plot[variable=\x,domain=1:1.5,samples=20,smooth] 
         ({\x},{-(\x*\x - 1)^(1/2)}); %% (y, z, x)
    \end{scope} 
\end{tikzpicture}
% }

%%%%%%%%%%%%%%%%%%%%%%%%%%%%%%%%%%%%%%%%%%%%%%%%%%%%%%%%%%%%%%%%%%%%%%%%%
%
%   Chapitre 5 - Fonctions
%
%%%%%%%%%%%%%%%%%%	Graphes des fonctions d'une variable 

% \newcommand{\trh}{
\begin{tikzpicture}[scale=1,font=\footnotesize]
\draw[->] (-1,0) -- (2.8,0) ;
\draw[->] (0,-0.5) -- (0,1.8) ;
\draw[thick] plot[domain=-0.5:2.5] (\x,{0.5+2.5*atan(\x-1)/180}) ;
\draw[thick] plot[domain=-0.5:2.5] (\x,{0.5+2.5*atan(\x-1)/180+0.5}) ;
\draw[dashed] (2,0) -- (2,1.15) -- (2,1.65) ;
\draw[->,thick] (2,1.15) -- (2,1.65) ;
\draw (2,1.4) node[right] {$b$} ;
\draw (2,-0.5) node{$x$} ;%(-0.5,1.15) node{$f(x)$} ;
\draw (2,1.12) node{$\bullet$} ;
\draw (2,1.12) node[right]{$f(x)$} ;
\draw (2,1.62) node{$\bullet$} ;
\draw (2,1.62) node[above]{$f(x)+b$} ;
\end{tikzpicture}
% }

% \newcommand{\trda}{
\begin{tikzpicture}[scale=1,font=\footnotesize]
\draw[->](-1.5,0) -- (4.5,0);
\draw[->](0,-0.5) -- (0,2);
\draw[domain=-1:1] plot (\x, {1/(1.5-\x*\x)});
\draw[thick](-1,0) -- (1,0);
\draw[thick] (0.5,0) node[below] {$D_f$} ;
\draw[thick](2,0) -- (4,0);
\draw[thick] (3,0) node[below] {$D_g=D_f+a$} ;
\end{tikzpicture}
% }


% \newcommand{\trdb}{
\begin{tikzpicture}[scale=1,font=\footnotesize]
\draw[->](-1.5,0) -- (4.5,0);
\draw[->](0,-0.5) -- (0,2);
\draw[domain=-1:1] plot (\x, {1/(1.5-\x*\x)});
\draw[thick, domain=2:4] plot (\x, {1/(1.5-(\x-3)*(\x-3))});
\draw (-1,0) -- (1,0);
%\draw (0.5,0) node[below] {$D_f$} ;
\draw (2,0) -- (4,0);
%\draw (3,0) node[below] {$D_g=D_f+a$} ;
\draw (2.5,0.8) node{$\bullet$} ;
\draw (2.5,0.8) node[above]{$g(x)=f(x-a)$} ;
\draw (-0.5,0.8) node{$\bullet$} ;
\draw (-0.5,0.8) node[left]{$f(x-a)$} ;
\draw[dashed] (-0.5,0.8)--(2.5,0.8);
\draw[dashed] (-0.5,0)--(-0.5,0.8);
\draw (-0.5,0) node[below]{$x-a$};
\draw[dashed] (2.5,0)--(2.5,0.8);
\draw (2.5,0) node[below]{$x$};
\end{tikzpicture}
% }

% \newcommand{\dilate}{
\begin{tikzpicture}[scale=1,font=\footnotesize]
\draw[->](-1.5,0) -- (2,0);
\draw[->](0,-0.5) -- (0,2);
\draw[domain=-1:1] plot (\x, {1/(1.5-\x*\x)});
\draw[thick, domain=-1:1] plot (\x, {2/(1.5-\x*\x)});
\draw (0,0.66) node{$\bullet$} ;
\draw (0,0.66) node[left]{$f(0)$} ;
\draw (0,1.32) node{$\bullet$} ;
\draw (0,1.32) node[left]{$g(0)=\mu f(0)$} ;
\end{tikzpicture}
% }

% \newcommand{\reflete}{
\begin{tikzpicture}[scale=1,font=\footnotesize]
\draw[->](-1.5,0) -- (2,0);
\draw[->](0,-1.5) -- (0,2);
\draw[domain=-1:1] plot (\x, {exp(\x)/2});
\draw[thick, domain=-1:1] plot (\x, {-exp(\x)/2});
%\draw (0,0.66) node{$\bullet$} ;
%\draw (0,0.66) node[left]{$f(0)$} ;
%\draw (0,1.32) node{$\bullet$} ;
%\draw (0,1.32) node[left]{$g(0)=\mu f(0)$} ;
\end{tikzpicture}
% }

% \newcommand{\dilateh}{
\begin{tikzpicture}[scale=0.5,font=\footnotesize]
\draw[->](-2.2*pi,0) -- (2.2*pi,0);
\draw[->](0,-1.5) -- (0,2);
\draw[domain=-2*pi:2*pi, samples=200] plot (\x, {sin(\x r)});
\draw[domain=-pi:pi, samples=200, thick] plot (\x, {sin(2*\x r)});
%\draw (0,0.66) node{$\bullet$} ;
%\draw (0,0.66) node[left]{$f(0)$} ;
%\draw (0,1.32) node{$\bullet$} ;
%\draw (0,1.32) node[left]{$g(0)=\mu f(0)$} ;
\end{tikzpicture}
% }


% \newcommand{\restreint}{
\begin{tikzpicture}[scale=0.5,font=\footnotesize]
\draw[->](-2.2*pi,0) -- (2.2*pi,0);
\draw[->](0,-1.5) -- (0,2);
\draw[domain=-2*pi:2*pi, samples=200] plot (\x, {cos(\x r)});
\draw[domain=-pi:pi, samples=200, thick] plot (\x, {cos(\x r)});
\draw[dashed, thick](-pi,-1.5)--(-pi,2);
\draw[dashed, thick](pi,-1.5)--(pi,2);
%\draw (0,0.66) node{$\bullet$} ;
%\draw (0,0.66) node[left]{$f(0)$} ;
%\draw (0,1.32) node{$\bullet$} ;
%\draw (0,1.32) node[left]{$g(0)=\mu f(0)$} ;
\end{tikzpicture}
% }

% \newcommand{\corestreint}{
\begin{tikzpicture}[scale=0.5,font=\footnotesize]
\draw[->](-2.2*pi,0) -- (2.2*pi,0);
\draw[->](0,-1.5) -- (0,2);
%\draw[domain=-2*pi:2*pi, samples=200] plot (\x, {cos(\x r)});
\draw[domain=-2*pi:2*pi, thick, samples=200] plot (\x, {cos(\x r)});
\draw[dashed, thick](-2*pi,1)--(2*pi,1);
\draw[dashed, thick](-2*pi,-1)--(2*pi,-1);
%\draw (0,0.66) node{$\bullet$} ;
%\draw (0,0.66) node[left]{$f(0)$} ;
%\draw (0,1.32) node{$\bullet$} ;
%\draw (0,1.32) node[left]{$g(0)=\mu f(0)$} ;
\end{tikzpicture}
% }

% \newcommand{\grapheF}{
\begin{tikzpicture}[scale=1,font=\footnotesize]
\draw[->] (-1,0) -- (2.8,0) ;
\draw[->] (0,-0.5) -- (0,1.8) ;
\draw[thick] plot[domain=-0.5:2.5] (\x,{0.5+2.5*atan(\x-1)/180}) ;
\draw[dashed] (2,0) -- (2,1.15) -- (0,1.15) ;
\draw (2,-0.5) node{$x$} (-0.5,1.15) node{$f(x)$} ;
\draw (2,1.5) node{$(x,f(x))$} (2,1.12) node{$\bullet$} ;
\draw (0.5,0.5) node{$\Gamma_f$} ;
\end{tikzpicture}
% }

% \newcommand{\grapheX}{
\begin{tikzpicture}[scale=.4,font=\footnotesize]
\draw[->] (-3,0) -- (3,0) ;
\draw[->] (0,-3) -- (0,3) ;
\draw[thick] plot[domain=-2.5:2.5] (\x,\x) ;
%\draw (3,-0.4) node{$x$} (0,3.3) node{$f(x)=x$} ;
\end{tikzpicture}
% }

% \newcommand{\grapheXX}{
\begin{tikzpicture}[scale=.4,font=\footnotesize]
\draw[->] (-3,0) -- (3,0) ;
\draw[->] (0,-3) -- (0,3) ;
\draw[thick] plot[domain=-1.7:1.7] (\x,{\x*\x}) ;
%\draw (3,-0.4) node{$x$} (0,3.3) node{$f(x)=x^2$} ;
\end{tikzpicture}
% }

% \newcommand{\grapheXXbis}{
\begin{tikzpicture}[scale=.4,font=\footnotesize]
\draw[->] (-3,0) -- (3,0) ;
\draw[->] (0,-3) -- (0,3) ;
\draw[thick] plot[domain=-1.7:1.7] (\x,{\x*\x}) ;
\draw (1.2,-0.4) node{$x$} (-2,3.3) node{$G$} ;
\multiput(10,-13)(0,5){8}{\line(0,1){2}}
\put(10,9){\circle*{2}}
\end{tikzpicture}
% }

% \newcommand{\grapheXXter}{
\begin{tikzpicture}[scale=.4,font=\footnotesize]
\draw[->] (-3,0) -- (3,0) ;
\draw[->] (0,0) -- (0,3) ;
\draw[thick] plot[domain=-1.7:1.7] (\x,{\x*\x}) ;
%\draw (1.2,-0.4) node{$x$} ;%(-2,3.3) node{$G$} ;
%\multiput(10,-13)(0,5){8}{\line(0,1){2}}
%\put(10,9){\circle*{2}}
\end{tikzpicture}
% }

% \newcommand{\grapheXXX}{
\begin{tikzpicture}[scale=.4,font=\footnotesize]
\draw[->] (-3,0) -- (3,0) ;
\draw[->] (0,-3) -- (0,3) ;
\draw[thick] plot[domain=-1.4:1.4] (\x,{\x*\x*\x}) ;
%\draw (3,-0.4) node{$x$} (0,3.3) node{$f(x)=x^3$} ;
\end{tikzpicture}
% }

% \newcommand{\grapheXXXX}{
\begin{tikzpicture}[scale=.4,font=\footnotesize]
\draw[->] (-3,0) -- (3,0) ;
\draw[->] (0,-3) -- (0,3) ;
\draw[thick] plot[domain=-1.3:1.3] (\x,{\x*\x*\x*\x}) ;
%\draw (3,-0.4) node{$x$} (0,3.3) node{$f(x)=x^4$} ;
\end{tikzpicture}
% }

% \newcommand{\grapheXXXXa}{
\begin{tikzpicture}[scale=.4,font=\footnotesize]
\draw[->] (-3,0) -- (3,0) ;
\draw[->] (0,-3) -- (0,3) ;
\draw[thick] plot[domain=0:1.3] (\x,{\x*\x*\x*\x}) ;
%\draw (3,-0.4) node{$x$} (0,3.3) node{$f(x)=x^4$} ;
\end{tikzpicture}
% }

% \newcommand{\grapheXXXXb}{
\begin{tikzpicture}[scale=.4,font=\footnotesize]
\draw[->] (-3,0) -- (3,0) ;
\draw[->] (0,-3) -- (0,3) ;
\draw[thick] plot[domain=-1.3:1.3] (\x,{\x*\x*\x*\x}) ;
%\draw (3,-0.4) node{$x$} (0,3.3) node{$f(x)=x^4$} ;
\end{tikzpicture}
% }

% \newcommand{\grapheXinv}{
\begin{tikzpicture}[scale=.21,font=\footnotesize]
\draw[->] (-6,0) -- (6,0) ;
\draw[->] (0,-6) -- (0,6) ;
\draw[thick] plot[domain=-5.6:-0.18] (\x,{1/\x}) ;
\draw[thick] plot[domain=0.18:5.6] (\x,{1/\x}) ;
%\draw (6,-0.8) node{$x$} (0,6.6) node{$f(x)=1/x$} ;
\end{tikzpicture}
% }

% \newcommand{\grapheXinvbis}{
\begin{tikzpicture}[scale=.21,font=\footnotesize]
\draw[->] (-6,0) -- (6,0) ;
\draw[->] (0,-6) -- (0,6) ;
\draw[thick] plot[domain=-5.6:-0.18] (\x,{1/\x}) ;
\draw[thick] plot[domain=0.18:5.6] (\x,{1/\x}) ;
\draw (1.2,-0.4) node{$x$} (1,6) node{$G$} ;
\multiput(10,-13)(0,5){8}{\line(0,1){2}}
\put(10,3.5){\circle*{2}}
\end{tikzpicture}
% }

% \newcommand{\grapheXXinv}{
\begin{tikzpicture}[scale=.21,font=\footnotesize]
\draw[->] (-6,0) -- (6,0) ;
\draw[->] (0,-6) -- (0,6) ;
\draw[thick] plot[domain=-5.6:-0.43] (\x,{1/(\x*\x)}) ;
\draw[thick] plot[domain=0.43:5.6] (\x,{1/(\x*\x)}) ;
%\draw (6,-0.8) node{$x$} (0,6.6) node{$f(x)=1/x^2$} ;
\end{tikzpicture}
% }

% \newcommand{\grapheXXXinv}{
\begin{tikzpicture}[scale=.21,font=\footnotesize]
\draw[->] (-6,0) -- (6,0) ;
\draw[->] (0,-6) -- (0,6) ;
\draw[thick] plot[domain=-5.6:-0.56] (\x,{1/(\x*\x*\x)}) ;
\draw[thick] plot[domain=0.56:5.6] (\x,{1/(\x*\x*\x)}) ;
\draw (6,-0.8) node{$x$} (0,6.6) node{$f(x)=1/x^3$} ;
\end{tikzpicture}
% }

% \newcommand{\grapheXXXXinv}{
\begin{tikzpicture}[scale=.21,font=\footnotesize]
\draw[->] (-6,0) -- (6,0) ;
\draw[->] (0,-6) -- (0,6) ;
\draw[thick] plot[domain=-5.6:-0.65] (\x,{1/(\x*\x*\x*\x)}) ;
\draw[thick] plot[domain=0.65:5.6] (\x,{1/(\x*\x*\x*\x)}) ;
\draw (6,-0.8) node{$x$} (0,6.6) node{$f(x)=1/x^4$} ;
\end{tikzpicture}
% }

% \newcommand{\grapheXsqrtII}{ 
\begin{tikzpicture}[scale=.4,font=\footnotesize]
\draw[->] (-3,0) -- (3,0) ;
\draw[->] (0,-1) -- (0,3) ;
\draw[thick] plot[domain=0:2.8] (\x,{sqrt(\x)}) ;
%\draw (3,-0.4) node{$x$} (0,3.3) node{$f(x)=\sqrt{x}$} ;
\end{tikzpicture}
% }

% \newcommand{\grapheXsqrtIII}{
\begin{tikzpicture}[scale=.4,font=\footnotesize]
\draw[->] (-3,0) -- (3,0) ;
\draw[->] (0,-1) -- (0,3) ;
\draw[thick] plot[domain=0:2.8] (\x,{sqrt(sqrt(\x))}) ;
\draw[thick] plot[domain=-2.8:0] (\x,{-sqrt(sqrt(-\x))}) ;
\draw (3,-0.4) node{$x$} (0,3.3) node{$f(x)=\sqrt[3]{x}$} ;
\end{tikzpicture}
% }

% \newcommand{\grapheXsqrtIV}{ 
\begin{tikzpicture}[scale=.4,font=\footnotesize]
\draw[->] (-3,0) -- (3,0) ;
\draw[->] (0,-1) -- (0,3) ;
\draw[thick] plot[domain=0:2.8] (\x,{sqrt(sqrt(sqrt(\x)))}) ;
\draw (3,-0.4) node{$x$} (0,3.3) node{$f(x)=\sqrt[4]{x}$} ;
\end{tikzpicture}
% }

% \newcommand{\grapheXsqrtV}{
\begin{tikzpicture}[scale=.4,font=\footnotesize]
\draw[->] (-3,0) -- (3,0) ;
\draw[->] (0,-1) -- (0,3) ;
\draw[thick] plot[domain=0:2.8] (\x,{sqrt(sqrt(sqrt(sqrt(\x))))}) ;
\draw[thick] plot[domain=-2.8:0] (\x,{-sqrt(sqrt(sqrt(sqrt(-\x))))}) ;
\draw (3,-0.4) node{$x$} (0,3.3) node{$f(x)=\sqrt[5]{x}$} ;
\end{tikzpicture}
% }

% \newcommand{\grapheXVA}{
\begin{tikzpicture}[scale=.4,font=\footnotesize]
\draw[->] (-3,0) -- (3,0) ;
\draw[->] (0,-1) -- (0,3) ;
\draw[thick] plot[domain=-2.5:2.5] (\x,{abs(\x)}) ;
%\draw (3,-0.4) node{$x$} (0,3.3) node{$f(x)=|x|$} ;
\end{tikzpicture}
% }

% \newcommand{\grapheXVAsmall}{
\begin{tikzpicture}[scale=.35,font=\footnotesize]
\draw[->] (-3,0) -- (3,0) ;
\draw[->] (0,-1) -- (0,3) ;
\draw[thick] plot[domain=-2.5:2.5] (\x,{abs(\x)}) ;
%\draw (2.7,-0.4) node{$x$} (-0.7,2.7) node{$|x|$} ;
\end{tikzpicture}
% }

% \newcommand{\grapheXXXVA}{
\begin{tikzpicture}[scale=.4,font=\footnotesize]
\draw[->] (-3,0) -- (3,0) ;
\draw[->] (0,-1) -- (0,3) ;
\draw[thick] plot[domain=-1.3:1.3] (\x,{abs(\x*\x*\x)}) ;
\draw (3,-0.4) node{$x$} (0,3.3) node{$f(x)=|x^3|$} ;
\end{tikzpicture}
% }

% \newcommand{\grapheXinvVA}{
\begin{tikzpicture}[scale=.21,font=\footnotesize]
\draw[->] (-6,0) -- (6,0) ;
\draw[->] (0,-2) -- (0,6) ;
\draw[thick] plot[domain=-5.6:-0.18] (\x,{abs(1/\x)}) ;
\draw[thick] plot[domain=0.18:5.6] (\x,{1/\x}) ;
\draw (6,-0.8) node{$x$} (0,6.6) node{$f(x)=|1/x|$} ;
\end{tikzpicture}
% }

% \newcommand{\grapheXsqrtIIIVA}{
\begin{tikzpicture}[scale=.4,font=\footnotesize]
\draw[->] (-3,0) -- (3,0) ;
\draw[->] (0,-3) -- (0,3) ;
\draw[thick] plot[domain=0:2.8] (\x,{sqrt(sqrt(\x))}) ;
\draw[thick] plot[domain=-2.8:0] (\x,{sqrt(sqrt(-\x))}) ;
\draw (3,-0.4) node{$x$} (0,3.3) node{$f(x)=|\sqrt[3]{x}|$} ;
\end{tikzpicture}
% }

% \newcommand{\grapheSIN}{
\begin{tikzpicture}[scale=.25,font=\footnotesize]
\draw[->] (-5,0) -- (5,0) ;
\draw[->] (0,-5) -- (0,5) ;
\draw[thick] plot[domain=-4.5:4.5] (\x,{2*sin(\x r)}) ;
%\draw (5,-0.6) node{$x$} (0,5.5) node{$f(x)=\sin(x)$} ;
\end{tikzpicture}
% }

% \newcommand{\grapheSINa}{
\begin{tikzpicture}[scale=.25,font=\footnotesize]
\draw[->] (-5,0) -- (5,0) ;
\draw[->] (0,-5) -- (0,5) ;
\draw[thick] plot[domain=0:3.1415] (\x,{2*sin(\x r)}) ;
%\draw (5,-0.6) node{$x$} (0,5.5) node{$f(x)=\sin(x)$} ;
\end{tikzpicture}
% }

% \newcommand{\grapheSINb}{
\begin{tikzpicture}[scale=.25,font=\footnotesize]
\draw[->] (-5,0) -- (5,0) ;
\draw[->] (0,-5) -- (0,5) ;
\draw[thick] plot[domain=-3.1415:3.1415] (\x,{2*sin(\x r)}) ;
%\draw (5,-0.6) node{$x$} (0,5.5) node{$f(x)=\sin(x)$} ;
\end{tikzpicture}
% }

% \newcommand{\grapheSINc}{
\begin{tikzpicture}[scale=.25,font=\footnotesize]
\draw[->] (-4,0) -- (7,0) ;
\draw[->] (0,-5) -- (0,5) ;
\draw[thick, domain=-pi:3*pi, samples=200] plot (\x,{2*sin(\x r)});
%\draw[thick] plot[domain=--3.1415:9.4247] (\x,{2*sin(deg(\x))}) ;
%\draw (5,-0.6) node{$x$} (0,5.5) node{$f(x)=\sin(x)$} ;
\end{tikzpicture}
% }

% \newcommand{\grapheCOS}{
\begin{tikzpicture}[scale=.25,font=\footnotesize]
\draw[->] (-5,0) -- (5,0) ;
\draw[->] (0,-5) -- (0,5) ;
\draw[thick] plot[domain=-4.5:4.5] (\x,{2*cos(\x r)}) ;
%\draw (5,-0.6) node{$x$} (0,5.5) node{$f(x)=\cos(x)$} ;
\end{tikzpicture}
% }

% \newcommand{\grapheTAN}{
\begin{tikzpicture}[scale=.25,font=\footnotesize]
\draw[->] (-5,0) -- (5,0) ;
\draw[->] (0,-5) -- (0,5) ;
\draw[thick] plot[domain=-1.36:1.36] (\x,{tan(\x r)}) ;
\draw[thick] plot[domain=1.78:4.5] (\x,{tan(\x r)}) ;
\draw[thick] plot[domain=-4.5:-1.78] (\x,{tan(\x r)}) ;
\draw[dashed] (-1.62,-4.8) -- (-1.62,4.8) ;
\draw[dashed] (1.62,-4.8) -- (1.62,4.8) ;
%\draw (5,-0.6) node{$x$} (0,5.5) node{$f(x)=\tan(x)$} ;
\end{tikzpicture}
% }

% \newcommand{\grapheCOT}{
\begin{tikzpicture}[scale=.25,font=\footnotesize]
\draw[->] (-5,0) -- (5,0) ;
\draw[->] (0,-5) -- (0,5) ;
\draw[thick] plot[domain=0.21:2.92] (\x,{cot(\x r)}) ;
\draw[thick] plot[domain=3.35:4.8] (\x,{cot(\x r)}) ;
\draw[thick] plot[domain=-0.21:-2.92] (\x,{cot(\x r)}) ;
\draw[thick] plot[domain=-3.35:-4.8] (\x,{cot(\x r)}) ;
\draw[dashed] (0,-4.8) -- (0,4.8) ;
\draw[dashed] (3.14,-4.8) -- (3.14,4.8) ;
\draw[dashed] (-3.14,-4.8) -- (-3.14,4.8) ;
\draw (5,-0.6) node{$x$} (0,5.5) node{$f(x)=\cot(x)$} ;
\end{tikzpicture}
% }

% \newcommand{\grapheASIN}{
\begin{tikzpicture}[scale=.4,font=\footnotesize]
\draw[->] (-3,0) -- (3,0) ;
\draw[->] (0,-2) -- (0,3) ;
\draw[thick] plot[domain=-1:1] (\x,{rad(asin(\x))}) ;
%\draw (3,-0.4) node{$x$} (0,3.3) node{$f(x)=\arcsin(x)$} ;
\end{tikzpicture}
% }

% \newcommand{\grapheACOS}{
\begin{tikzpicture}[scale=.4,font=\footnotesize]
\draw[->] (-3,0) -- (3,0) ;
\draw[->] (0,-2) -- (0,3) ;
\draw[thick] plot[domain=-1:1] (\x,{rad(acos(\x))}) ;
%\draw (3,-0.4) node{$x$} (0,3.3) node{$f(x)=\arccos(x)$} ;
\end{tikzpicture}
% }

% \newcommand{\grapheATAN}{
\begin{tikzpicture}[scale=.4,font=\footnotesize]
\draw[->] (-3,0) -- (3,0) ;
\draw[->] (0,-2) -- (0,3) ;
\draw[thick] plot[domain=-2.9:2.9] (\x,{rad(atan(\x))}) ;
\draw[dashed] (-3,1.5) -- (3,1.5) ;
\draw[dashed] (-3,-1.5) -- (3,-1.5) ;
%\draw (3,-0.4) node{$x$} (0,3.3) node{$f(x)=\arctan(x)$} ;
\end{tikzpicture}
% }

% \newcommand{\grapheACOT}{
\begin{tikzpicture}[scale=.4,font=\footnotesize]
\draw[->] (-3,0) -- (3,0) ;
\draw[->] (0,-2) -- (0,3) ;
\draw[thick] plot[domain=-2.9:-0.01] (\x,{rad(atan(1/\x))+3.14}) ;
\draw[thick] plot[domain=0.01:2.9] (\x,{rad(atan(1/\x))}) ;
\draw (3,-0.4) node{$x$} (0,3.3) node{$f(x)=\mathrm{arccot}(x)$} ;
\end{tikzpicture}
% }

% \newcommand{\grapheEXP}{
\begin{tikzpicture}[scale=.4,font=\footnotesize]
\draw[->] (-3,0) -- (3,0) ;
\draw[->] (0,-1) -- (0,3) ;
\draw[thick] plot[domain=-2.9:1.05] (\x,{exp(\x)}) ;
%\draw (3,-0.4) node{$x$} (0,3.4) node{$f(x)=\exp(x)$} ;
\end{tikzpicture}
% }

% \newcommand{\grapheEXPinv}{
\begin{tikzpicture}[scale=.4,font=\footnotesize]
\draw[->] (-3,0) -- (3,0) ;
\draw[->] (0,-1) -- (0,3) ;
\draw[thick] plot[domain=-1.05:2.9] (\x,{exp(-\x)}) ;
\draw (3,-0.4) node{$x$} (0,3.4) node{$f(x)=\exp(-x)=\frac{1}{\exp(x)}$} ;
\end{tikzpicture}
% }

% \newcommand{\grapheLN}{
\begin{tikzpicture}[scale=.4,font=\footnotesize]
\draw[->] (-3,0) -- (3,0) ;
\draw[->] (0,-3) -- (0,3) ;
\draw[thick] plot[domain=0.08:2.9] (\x,{ln(\x)}) ;
%\draw (3,-0.4) node{$x$} (0,3.4) node{$f(x)=\ln(x)$} ;
\end{tikzpicture}
% }

% \newcommand{\grapheLNinv}{
\begin{tikzpicture}[scale=.4,font=\footnotesize]
\draw[->] (-3,0) -- (3,0) ;
\draw[->] (0,-3) -- (0,3) ;
\draw[thick] plot[domain=0.08:2.9] (\x,{-ln(\x)}) ;
\draw (3,-0.4) node{$x$} (0,3.4) node{$f(x)=-\ln(x)=\ln\left(\frac{1}{x}\right)$} ;
\end{tikzpicture}
% }

% \newcommand{\grapheSINH}{
\begin{tikzpicture}[scale=.4,font=\footnotesize]
\draw[->] (-3,0) -- (3,0) ;
\draw[->] (0,-3) -- (0,3) ;
\draw[thick] plot[domain=-2.5:2.5] (\x,{0.5*sinh(\x)}) ;
%\draw (3,-0.4) node{$x$} (0,3.4) node{$f(x)=\sinh(x)$} ;
\end{tikzpicture}
% }

% \newcommand{\grapheCOSH}{
\begin{tikzpicture}[scale=.4,font=\footnotesize]
\draw[->] (-3,0) -- (3,0) ;
\draw[->] (0,-3) -- (0,3) ;
\draw[thick] plot[domain=-2.5:2.5] (\x,{0.5*cosh(\x)}) ;
%\draw (3,-0.4) node{$x$} (0,3.4) node{$f(x)=\cosh(x)$} ;
\end{tikzpicture}
% }

% \newcommand{\grapheTANH}{
\begin{tikzpicture}[scale=.4,font=\footnotesize]
\draw[->] (-3,0) -- (3,0) ;
\draw[->] (0,-3) -- (0,3) ;
\draw[thick] plot[domain=-2.5:2.5] (\x,{tanh(\x)}) ;
\draw[dashed] (-3,1) -- (3,1) ;
\draw[dashed] (-3,-1) -- (3,-1) ;
\draw (3,-0.4) node{$x$} (0,3.4) node{$f(x)=\tanh(x)$} ;
\end{tikzpicture}
% }

%%%%%%%%%%%%%%%%%%      Croissance

% \newcommand{\croissant}{
\begin{tikzpicture}[scale=.4,font=\footnotesize]
\draw[->] (-0.7,0) -- (1.5,0) ;
\draw[->] (0,-0.5) -- (0,1.5) ;
\draw[thick] plot[domain=-0.6:1.2] (\x,{(\x-0.3)^3+0.3}) ;
\end{tikzpicture}
% }

% \newcommand{\decroissant}{
\begin{tikzpicture}[scale=.4,font=\footnotesize]
\draw[->] (-0.7,0) -- (1.5,0) ;
\draw[->] (0,-0.5) -- (0,1.5) ;
\draw[thick] plot[domain=-0.6:1.2] (\x,{-(\x-0.3)^3+0.3}) ;
\end{tikzpicture}
% }

% \newcommand{\constant}{
\begin{tikzpicture}[scale=.4,font=\footnotesize]
\draw[->] (-0.7,0) -- (2.5,0) ;
\draw[->] (0,-0.5) -- (0,2.2) ;
\draw[thick] plot[domain=-0.6:1.2] (\x,{0.7}) ;
\draw[thick] plot[domain=1.2:2] (\x,{0.7+(\x-1.2)^2}) ;
\end{tikzpicture}
% }

%%%%%%%%%%%%%%%%%%      Convexité

% \newcommand{\convexe}{
\begin{tikzpicture}[scale=.4,font=\footnotesize]
%\draw[->] (-0.7,0) -- (1.5,0) ;
%\draw[->] (0,-0.5) -- (0,1.5) ;
%\draw[thick] plot[domain=-0.6:1.2] (\x,{(\x-0.3)^2+0.3}) ;
\draw[->] (-2,0) -- (2,0);
\draw[->] (0,0) -- (0,2.2);
\draw[thick, domain=-2:2.1] plot(\x, {\x*\x/2+1});
\draw (-1,1.5) node{$\bullet$}; 
\draw (-1,1.5) node[below]{$A$} ;
\draw (0.5,1.1) node{$\bullet$}; 
\draw (0.5,1.1) node[below]{$B$} ;
\draw (-1,1.5) -- (0.5,1.1) ;
\end{tikzpicture}
% }

% \newcommand{\concave}{
\begin{tikzpicture}[scale=.4,font=\footnotesize]
%\draw[->] (-0.7,0) -- (1.5,0) ;
%\draw[->] (0,-0.5) -- (0,1.5) ;
%\draw[thick] plot[domain=-0.6:1.2] (\x,{(\x-0.3)^2+0.3}) ;
\draw[->] (-2,0) -- (2,0);
\draw[->] (0,0) -- (0,2.2);
\draw[thick, domain=-2:2.1] plot(\x, {-\x*\x/2+1});
\draw (-1,0.5) node{$\bullet$}; 
\draw (-1,0.5) node[left]{$A$} ;
\draw (0.5,0.9) node{$\bullet$}; 
\draw (0.5,0.9) node[below]{$B$} ;
\draw (-1,0.5) -- (0.5,0.9) ;
\end{tikzpicture}
% }

%%%%%%%%%%%%%%%%%%      Parité

% \newcommand{\pair}{
\begin{tikzpicture}[scale=.5,font=\footnotesize]
\draw[->] (-2.5,0) -- (2.8,0) ;
\draw[->] (0,-0.5) -- (0,1.5) ;
\draw[thick] plot[domain=-2.5:2.5,smooth,samples=50] (\x,{cos((1.2*\x)^4 r)+0.1}) ;
\end{tikzpicture}
% }

% \newcommand{\impair}{
\begin{tikzpicture}[scale=.5,font=\footnotesize]
\draw[->] (-2.5,0) -- (2.8,0) ;
\draw[->] (0,-1) -- (0,1.5) ;
\draw[thick] plot[domain=-2.5:2.5,smooth,samples=50] (\x,{sin((1.2*\x)^3 r)}) ;
\end{tikzpicture}
% }

% \newcommand{\periodique}{
\begin{tikzpicture}[scale=.5,font=\footnotesize]
\draw[->] (-3,0) -- (4.2,0) ;
\draw[->] (0,-1.5) -- (0,1.5) ;
\draw[thick] plot[domain=-3:4,smooth,samples=50] 
(\x,{0.5*sin(10*\x r)+0.5*cos(12*\x r)}) ;
\end{tikzpicture}
% }

%%%%%%%%%%%%%%%%%    Exos sur les graphes

% \newcommand{\exoLNA}{
\begin{tikzpicture}[scale=.5,font=\footnotesize]
\draw[->] (-3,0) -- (3,0) ;
\draw[->] (0,-3) -- (0,3) ;
\draw[thick] plot[domain=0.1:2.8] (\x,{ln(\x)}) ;
\draw (1,0) node{$\bullet$} (1,-0.5) node{$1$} ;
\draw (3,-0.4) node{$x$} (0,3.4) node{$\ln(x)$} ;
\end{tikzpicture}
% }

% \newcommand{\exoLNB}{
\begin{tikzpicture}[scale=.5,font=\footnotesize]
\draw[->] (-3,0) -- (3,0) ;
\draw[->] (0,-3) -- (0,3) ;
\draw[thick] plot[domain=0.1:2.8] (\x,{2*ln(\x)+1}) ;
\draw (1,0) node{$|$} (1,-0.7) node{$1$} ;
\draw (0,1) node{$-$} (-0.5,1) node{$1$} ;
\draw (3,-0.4) node{$x$} (0,3.4) node{$f(x)=2\ln(x)+1$} ;
\draw (1,1) node{$\bullet$} ;
\end{tikzpicture}
% }

% \newcommand{\exoCOSA}{
\begin{tikzpicture}[scale=.25,font=\footnotesize]
\draw[->] (-8,0) -- (8,0) ;
\draw[->] (0,-5) -- (0,5) ;
\draw (0,2) node{$\bullet$} (-0.6,2.3) node{$1$} ;
\draw (3.14,0) node{$|$} (3.14,1) node{$\pi$} ;
\draw (6.28,0) node{$|$} (6.28,-1) node{$2\pi$} ;
\draw[thick,smooth,samples=100] plot[domain=-7.5:7.5] (\x,{2*cos(\x r)}) ;
\draw (8,-0.6) node{$x$} (0,5.5) node{$\cos(x)$} ;
\end{tikzpicture}
% }

% \newcommand{\exoCOSB}{
\begin{tikzpicture}[scale=.25,font=\footnotesize]
\draw[->] (-8,0) -- (8,0) ;
\draw[->] (0,-5) -- (0,5) ;
\draw (0,2) node{$-$} (-0.6,2.3) node{$1$} ;
\draw (3.14,0) node{$|$} (3.14,1) node{$\pi$} ;
\draw (6.28,0) node{$|$} (6.28,1) node{$2\pi$} ;
\draw[thick,smooth,samples=100] plot[domain=-7.5:7.5] (\x,{2*cos(2*\x r)-2}) ;
\draw (8,-0.6) node{$\theta$} (0,5.5) node{$u(\theta)=\cos(2\theta)-1$} ;
\end{tikzpicture}
% }

% \newcommand{\exoSQRTA}{ 
\begin{tikzpicture}[scale=.4,font=\footnotesize]
\draw[->] (-1,0) -- (3,0) ;
\draw[->] (0,-2) -- (0,2) ;
\draw[thick] plot[domain=0:2.8] (\x,{sqrt(\x)}) ;
\draw (3,-0.4) node{$t$} (0,2.3) node{$\sqrt{t}$} ;
\end{tikzpicture}
% }

% \newcommand{\exoSQRTB}{ 
\begin{tikzpicture}[scale=.4,font=\footnotesize]
\draw[->] (-1,0) -- (4,0) ;
\draw[->] (0,-2) -- (0,2) ;
\draw[thick] plot[domain=1:3.8] (\x,{sqrt(\x-1)}) ;
\draw (4,-0.4) node{$t$} (0,2.3) node{$\sqrt{t-1}$} ;
\end{tikzpicture}
% }

% \newcommand{\exoSQRTC}{ 
\begin{tikzpicture}[scale=.4,font=\footnotesize]
\draw[->] (-1,0) -- (4,0) ;
\draw[->] (0,-2) -- (0,2) ;
\draw[thick] plot[domain=1:3.8] (\x,{-sqrt(\x-1)}) ;
\draw (4,-0.4) node{$t$} (0,2.3) node{$z(t)=-\sqrt{t-1}$} ;
\end{tikzpicture}
% }

%%%%%%%%%%%%%%%%%%     Réciproques

% \newcommand{\grapheFF}{
\begin{tikzpicture}[scale=.4,font=\footnotesize]
\draw[->] (-3,0) -- (3,0) ;
\draw[->] (0,-2.5) -- (0,2.5) ;
\draw[thick] plot[domain=-1.3:1] (\x,{tan(\x r)+1}) ;
\draw[dashed] plot[domain=-2:2] ({\x},{\x}) ;
\draw (2.8,-0.5) node{$x$} (-1,2.3) node{$f(x)$} ;
\end{tikzpicture}
% }

% \newcommand{\grapheFFINV}{
\begin{tikzpicture}[scale=.4,font=\footnotesize]
\draw[->] (-3,0) -- (3,0) ;
\draw[->] (0,-2.5) -- (0,2.5) ;
\draw[thick] plot[domain=-2.9:2.9] (\x,{rad(atan(\x-1))}) ;
\draw[dashed] plot[domain=-2:2] ({\x},{\x}) ;
\draw (2.8,-0.5) node{$y$} (-1.5,2.3) node{$f^{-1}(y)$} ;
\end{tikzpicture}
% }

% \newcommand{\grapheXXTOT}{
\begin{tikzpicture}[scale=.4,font=\footnotesize]
\draw[->] (-2.5,0) -- (2.5,0) ;
\draw[->] (0,-.5) -- (0,2) ;
\draw[thick] plot[domain=-1.4:1.4] (\x,{\x*\x}) ;
\draw (2.5,-0.4) node{$x$} (-0.6,1.8) node{$x^2$} ;
\end{tikzpicture}
% }

% \newcommand{\grapheXXRED}{
\begin{tikzpicture}[scale=.4,font=\footnotesize]
\draw[->] (-2.5,0) -- (2.5,0) ;
\draw[->] (0,-.5) -- (0,2) ;
\draw[thick] plot[domain=0:1.3] (\x,{\x*\x}) ;
\draw (2.5,-0.4) node{$x$} (-0.6,1.8) node{$x^2$} ;
\end{tikzpicture}
% }

% \newcommand{\grapheSQRT}{ 
\begin{tikzpicture}[scale=.4,font=\footnotesize]
\draw[->] (-2.5,0) -- (2.5,0) ;
\draw[->] (0,-.5) -- (0,2) ;
\draw[thick] plot[domain=0:2.8] (\x,{sqrt(\x)}) ;
\draw (2.5,-0.4) node{$y$} (-0.8,1.8) node{$\sqrt{y}$} ;
\end{tikzpicture}
% }

% \newcommand{\grapheSINTOT}{
\begin{tikzpicture}[scale=.4,font=\footnotesize]
\draw[->] (-3,0) -- (3,0) ;
\draw[->] (0,-1.5) -- (0,1.5) ;
\draw[thick] plot[domain=-3:3] (\x,{sin(\x r)}) ;
\draw (2.8,-0.6) node{$x$} (-1,1.5) node{$\sin x$} ;
\end{tikzpicture}
% }

% \newcommand{\grapheSINRED}{
\begin{tikzpicture}[scale=.4,font=\footnotesize]
\draw[->] (-2.5,0) -- (2.5,0) ;
\draw[->] (0,-1.5) -- (0,1.5) ;
\draw[thick] plot[domain=-1.57:1.57] (\x,{sin(\x r)}) ;
\draw (2.8,-0.6) node{$x$} (-1,1.5) node{$\sin x$} ;
\end{tikzpicture}
% }

% \newcommand{\grapheASINRED}{
\begin{tikzpicture}[scale=.4,font=\footnotesize]
\draw[->] (-2,0) -- (2,0) ;
\draw[->] (0,-1.5) -- (0,1.5) ;
\draw[thick] plot[domain=-1:1] (\x,{rad(asin(\x))}) ;
\draw (1.8,-0.4) node{$y$} (-2,1.5) node{$\arcsin y$} ;
\end{tikzpicture}
% }







%%%%%%%%%%%%%%%##################%%%%%%%%%%%%%%%%%%%%%%%%%%%%#####################
\vspace{2cm}
\vspace{2cm}
\vspace{2cm}
\vspace{2cm}
\vspace{2cm}
\vspace{2cm}
\vspace{2cm}



$$
\grapheF
$$


%%%%%%%%%%%%%%%%%%%%%%%%%%%%%%%%%%%%%%%%%%%

\grapheX \quad \grapheXX \quad \grapheXXX %\quad \grapheXXXX

\grapheXinv \quad \grapheXXinv \quad \grapheXXXinv %\quad \grapheXXXXinv 

\grapheXsqrtII \quad \grapheXsqrtIII \quad \grapheXsqrtIV %\quad \grapheXsqrtV

%%%%%%%%%%%%%%%%%%%%%%%%%%%%%%%%%%%%%%%%%%%

\begin{frame}[plain]
\only<1>{\addtocounter{framenumber}{-1}}
\frametitle{\bf D'autres graphes \`a conna\^{\i}tre !} 
\medskip 

\grapheXVA \quad \grapheXXXVA \quad \grapheXinvVA \quad %\grapheXsqrtIIIVA 

\grapheSIN \quad \grapheCOS \quad \grapheTAN %\quad \grapheCOT

\grapheASIN \quad \grapheACOS \quad \grapheATAN %\quad \grapheACOT

\end{frame}

%%%%%%%%%%%%%%%%%%%%%%%%%%%%%%%%%%%%%%%%%%%

\begin{frame}
\frametitle{\bf D'autres encore... ouf !} 
\medskip 

\qquad \qquad \grapheEXP \qquad \grapheEXPinv 

\qquad \qquad \grapheLN \qquad \grapheLNinv

\grapheSINH \qquad \grapheCOSH \qquad \grapheTANH

\end{frame}

%%%%%%%%%%%%%%%%%%%%%%%%%%%%%%%%%%%%%%%%%%%%%%%%%%
%%%%%%%%%%%%%%%%%%%%%%%%%%%%%%%%%%%%%%%%%%%%%%%%%%

\raisebox{-.7\height}{\croissant}  

\raisebox{-.7\height}{\decroissant}  

\raisebox{-.7\height}{\constant}  

%%%%%%%%%%%%%%%%%%%%%%%%%%%%%%%%%%%%%%%%%%%%%%%%%%

\begin{frame}[plain]
\only<1>{\addtocounter{framenumber}{-1}}
\frametitle{\bf Fonctions convexes et concaves}
\medskip 

{\small 
La deuxi\`eme propri\'et\'e qu'on voit sur le graphe est la 
\underline{convexit\'e}.}
\vspace*{2mm}

{\bf D\'efinition:} 
Soit $f:\R\rightarrow\R$ une fonction et $D\subset D_f$. On dit que: 
\vspace*{1mm}

\begin{itemize}
\bitem
\parbox[t]{6cm}{
$f$ est {\bf convexe sur $D$} si elle a la forme 
}
\quad
\raisebox{-.5\height}{\convexe}  
\vspace*{1mm}

\bitem
\parbox[t]{6cm}{
$f$ est {\bf concave sur $D$} si elle a la forme
}
\quad
\raisebox{-.5\height}{\concave} 
\vspace*{1mm}

\bitem
\parbox[t]{6cm}{
$f$ est {\bf plate sur $D$} si elle est constante
}
\quad
\raisebox{-.5\height}{\constant}  
\end{itemize} 
\vspace*{1mm}

\pause
{\small
Si on n'indique pas l'ensemble $D$, on sous-entend qu'on parle de tout le 
domaine de d\'efinition $D_f$.
} 
\vspace*{2mm}

\pause
{\small 
{\bf Exemples:}
\begin{itemize}
\bitem
Les polyn\^omes $x^n$ et les fractions $\frac{1}{x^n}$ sont convexes si $n$ 
est pair. 
\vspace*{1mm}

Si $n$ est impair, ils sont concaves pour $x<0$ et convexes pour $x>0$.  
\vspace*{1mm}

\bitem
Les racines $\sqrt[k]{x}$ sont concaves. 
\vspace*{1mm}

\bitem
L'exponentiel $e^x$ est convexe. Le logarithme $\ln x$ est concave. 
\end{itemize} 
}

\end{frame}

%%%%%%%%%%%%%%%%%%%%%%%%%%%%%%%%%%%%%%%%%%%%%%%%%%

\raisebox{-.8\height}{\pair}
\raisebox{-.8\height}{\impair} 
\raisebox{-.8\height}{\periodique} 

%%%%%%%%%%%%%%%%%%%%%%%%%%%%%%%%%%%%%%%%%%%%%%%%%%

\raisebox{-\height}{\exoLNA} \qquad\qquad 
\raisebox{-\height}{\exoLNB}


\raisebox{-\height}{\exoSQRTA} \qquad
\raisebox{-\height}{\exoSQRTB} \qquad
\raisebox{-\height}{\exoSQRTC} 

%%%%%%%%%%%%%%%%%%%%%%%%%%%%%%%%%%%%%%%%%%%%%%%%%%
$$
\grapheFF \qquad \grapheFFINV
$$

%%%%%%%%%%%%%%%%%%%%%%%%%%%%%%%%%%%%%%%%%%%%%%%%%%

Ex.:\quad 
\raisebox{-.5\height}{\grapheXXTOT \quad \grapheXXRED \quad \grapheSQRT}
Ex.:\quad 
\raisebox{-.5\height}{\grapheSINTOT \quad \grapheSINRED \quad \grapheASINRED}

%%%%%%%%%%%%%%%%%%%%%%%%%%%%%%%%%%%%%%%%%%%%%%%%%%


\end{document}